\documentclass[12pt]{exam}
%\documentclass[10pt,answers]{exam}

\usepackage[OT2]{fontenc}
\usepackage[utf8x]{inputenc}
\usepackage[serbian]{babel}
\usepackage{amsmath,amssymb}
\usepackage{geometry}

\ifprintanswers
	\geometry{a5paper,margin=1.5cm}
\else
	\geometry{a5paper,margin=1.5cm}
\fi

% Замена наслова за решење
\renewcommand{\solutiontitle}{\noindent Концепт:\enspace}
\cfoot[]{Страница \thepage\ од \numpages}

\begin{document}

\ifprintanswers\else
\section*{Комбинације за усмени део испита}
\fi

% --------------------------
\subsection*{Комбинација 1}
\begin{questions}
\question Објасни формулу за квадрат бинома. Наведи доказ и илуструј на примеру.
\begin{solution}
$(a\pm b)^2=a^2\pm 2ab+b^2$.  
Доказ преко развијања производа и преко геометријске интерпретације (плочице/површине).  
Пример: $(3x-2)^2=9x^2-12x+4$.

Смернице за потпитања:
  \par Шта значи „бином“? 
  \par Како бисмо проверили тачност са конкретним бројевима?
  \par Зашто је корисно знати ову формулу уместо увек множити?
\end{solution}

\question Објасни шта су централни и периферијски углови у кругу. Каква је њихова веза?
\begin{solution}
Централни угао има врх у центру круга, периферијски на кружници.  
Периферијски угао над истим луком је половина централног.  
Илустрација скицом.

Смернице за потпитања:
  \par Можеш ли навести пример из праксе? 
  \par Шта се дешава ако лук захвата пола круга?
  \par Да ли је могуће да централни угао буде мањи од периферијског над истим луком?
\end{solution}

\question Растави на чиниоце $9x^2-25$.
\begin{solution}
Формула за разлику квадрата: $a^2-b^2=(a-b)(a+b)$.  
$9x^2-25=(3x-5)(3x+5)$.

Смернице за потпитања:
  \par Која је улога препознавања „потпуног квадрата“?
  \par Шта би било са $9x^2+25$? Да ли се може раставити преко реалних бројева?
\end{solution}
\end{questions}

\ifprintanswers\newpage\else\fi

% --------------------------
\subsection*{Комбинација 2}
\begin{questions}
\question Објасни поступак растављања полинома на чиниоце. Зашто је тај поступак важан?
\begin{solution}
Коришћење заједничког чиниоца, формула $(a\pm b)^2$, $a^2-b^2$, груписање.  
Важност: скраћивање разломака, решавање једначина, поједностављивање израза.

Смернице за потпитања:
  \par Када би најпре требало тражити заједнички чинилац?
  \par Можеш ли навести пример из алгебарске једначине?
\end{solution}

\question Шта значи „уписана“ и „описана“ кружница троугла?
\begin{solution}
Уписана кружница је тангентна на све странице троугла; описана кружница пролази кроз све темена.  
Веза са центром уписане кружнице (пресек симетрала углова) и центром описане кружнице (пресек симетрала страница).

Смернице за потпитања:
  \par Како бисмо конструктивно нашли центар уписане/описане кружнице? 
  \par Да ли је увек могуће у троугао уписати или око њега описати кружницу?
\end{solution}

\question Израчунај површину кружног исечка ако је $r=9\,\mathrm{cm}$ и угао $60^\circ$.
\begin{solution}
$P=\dfrac{\alpha}{360^\circ}\cdot\pi r^2$.
$$
 P = \frac{60}{360}\pi\cdot 81
 = \frac{81}{6}\pi
 = 13{,}5\pi\ \mathrm{cm}^2
.$$

Смернице за потпитања:
  \par Шта ако је угао $360^\circ$? 
  \par Да ли је површина директно пропорционална углу?
\end{solution}
\end{questions}

\ifprintanswers\newpage\else\newpage\fi

% --------------------------
\subsection*{Комбинација 3}
\begin{questions}
\question Како се рачуна обим и површина круга? Објасни ако је познат полупречник.
\begin{solution}
$O=2\pi r$, $P=\pi r^2$.
Геометријско оправдање преко „приближавања“ кругу мноштва правилних многоуглова.

Смернице за потпитања:
  \par Шта ако је дат пречник уместо полупречника?
  \par Шта се дешава са обимом и површином ако удвостручимо $r$?
\end{solution}

\question Објасни конструкцију тангенте на кружницу из задате
спо\-ља\-шње тачке.
\begin{solution}
 
1. Спојити центар $O$ и тачку $P$.  
2. Наћи средиште $S$ дужи $OP$.  
3. Описати кружницу $k(S,SO)$ која сече основну кружницу у додирним тачкама $T_1,T_2$.  
4. Тангенте су $PT_1$, $PT_2$.  

Смернице за потпитања:
  \par Зашто је угао $OT_1P$ прав? (Талесова теорема)
  \par Може ли из унутрашње тачке постојати тангента?
\end{solution}

\question Израчунај дужину тетиве у кругу полупречника $r=13\,\mathrm{cm}$ чије је средиште удаљено $5\,\mathrm{cm}$ од центра.
\begin{solution}
$x=\sqrt{r^2-d^2}=\sqrt{13^2-5^2}=\sqrt{169-25}=12$.  
$AB=2x=24\,\mathrm{cm}$.

Смернице за потпитања:
  \par Како се формула добија из Питагорине теореме?
  \par Шта би било да је $d>r$?
\end{solution}
\end{questions}

\ifprintanswers\newpage\else\fi

% --------------------------
\subsection*{Комбинација 4}
\begin{questions}
\question Објасни разлику између алгебарског израза и једначине. 
\begin{solution}
Израз даје вредност за дате променљиве; једначина садржи знак једнакости и решава се за непознату.  
Примери: $3x^2-2x+5$ (израз), $3x^2-2x+5=0$ (једначина).

Смернице за потпитања:
  \par Шта је полином? 
  \par Како знамо колико решења може имати једначина другог степена?
\end{solution}

\question Објасни конструкцију уписане кружнице у троугао.
\begin{solution}
Центар уписане кружнице је пресек симетрала углова. Ради се конструкција три симетрала, њихов пресек је центар. Полупречник је растојање од центра до неке странице.

Смернице за потпитања:
  \par Зашто симетрала углова, а не страница?
  \par Може ли се конструкција извести само помоћу шестара и лењира?
\end{solution}

\question Израчунај површину круга полупречника $6\,\mathrm{cm}$.
\begin{solution}
$P=\pi r^2=36\pi\,\mathrm{cm}^2$.

Смернице за потпитања:
  \par Шта би било ако се рачуна приближно ($\pi\approx3{,}14$)?
  \par Како се формула за површину круга може извести?
\end{solution}
\end{questions}

\ifprintanswers\newpage\else\fi

% --------------------------
\subsection*{Комбинација 5}
\begin{questions}
\question Шта је разлика квадрата и како се примењује у растављању израза?
\begin{solution}
$a^2-b^2=(a-b)(a+b)$.  
Примена: скраћивање, решавање једначина. Геометријска интерпретација (разлика површина квадрата).

Смернице за потпитања:
  \par Како би изгледала слична формула за $a^2+2ab+b^2$?
  \par Може ли $a^2+b^2$ да се растави?
\end{solution}

\question Објасни шта је круг, а шта кружница. Које су главне величине које уводимо уз њих?
\begin{solution}
Круг је скуп свих тачака у равни чије је растојање од центра највише $r$.  
Кружница је гранична линија (растојање тачно $r$).  
Главне величине: пречник, полупречник, обим, површина.

Смернице за потпитања:
  \par Која је разлика у димензијама (круг – површина, кружница – линија)?
  \par Да ли је пречник увек два пута дужи од полупречника?
\end{solution}

\question Израчунај дужину лука у кругу полупречника $10\,\mathrm{cm}$ за угао $72^\circ$.
\begin{solution}
$l=\dfrac{\alpha}{360^\circ}\cdot 2\pi r=\frac{72}{360}\cdot 20\pi=4\pi\,\mathrm{cm}$.

Смернице за потпитања:
  \par Шта ако је угао $180^\circ$? 
  \par Како изгледа зависност дужине лука од полупречника?
\end{solution}
\end{questions}

\end{document}
