%\documentclass[11pt,a5paper,twoside,addpoints,noanswers]{exam} % задаци
\documentclass[11pt,a5paper,twoside,addpoints,answers]{exam} % решења

\usepackage[OT2]{fontenc}
\usepackage[utf8x]{inputenc}
\usepackage[serbian]{babel}
\usepackage{multicol}
\usepackage{amssymb,amsmath}
\usepackage{graphicx}
\usepackage{geometry}
\geometry{a5paper, margin=1.5cm}

% Макро за јединице мере
\newcommand{\measure}[2]{#1\,\mathrm{#2}}

% Макро за варијанте
\newcommand{\variant}[3]{#1}

% Подешавања за exam
%\printanswers % <- откоментаришите да бисте видели решења
\renewcommand{\solutiontitle}{\noindent\textrm{Решење:}\enspace}
\pointsinmargin
\pointname{}
\hqword{Задатак:}
\hpgword{Страница:}
\hpword{Поени:}
\hsword{Остварено:}
\htword{Збир}
\vqword{Задатак:}
\vpgword{Страница:}
\vpword{Поени:}
\vsword{Остварено:}
\vtword{Збир}
\cellwidth{1em}
\cfoot[]{Страница \thepage\ од \numpages}
\addto{\captionsserbian}{\renewcommand{\abstractname}{Упут{}ство}}

\title{Припремна настава}
\author{$\mathrm{VII}_\Box$ Полиноми 2/2. Круг
 \thanks{
  29 одлично,
  22 врло добро,
  15 добро,
   8 довољно.
 }
}
\date{Станишић, 14.\ август 2025.}

\pagestyle{headandfoot}
\runningheader{Полиноми 2/2. Круг}
	{}{варијанта \variant 123}
\runningfooter{четвртак}
	{}{Страна \thepage\ од \numpages}

\begin{document}
\maketitle
\thispagestyle{headandfoot}

\ifprintanswers\else
\begin{flushleft}\scriptsize
\gradetable[h]
\end{flushleft}
\fi

\begin{abstract}
Пажљиво прочитај сваки задатак. У поступку пиши све потребне рачуне. Решења пиши у сређеном облику.
\end{abstract}

\begin{questions}

% =========================
% I. ПОЛИНОМИ – 2. део
% =========================
\section*{Полиноми — део 2}

\question[3]
Разложи на факторе:
\[
\variant{
x^2+5x+6
}{
a^2-9
}{
m^2+7m+10
}.
\]
\begin{solution}[\stretch 2]
\variant{
$(x+2)(x+3)$
}{
$(a-3)(a+3)$
}{
$(m+2)(m+5)$
}.
\end{solution}

\question[3]
Разложи на факторе:
\[
\variant{
4x^2-12x
}{
y^2+6y
}{
5p^2-15p
}.
\]
\begin{solution}[\stretch 2]
\variant{
$4x(x-3)$
}{
$y(y+6)$
}{
$5p(p-3)$
}.
\end{solution}

\question[4]
Скрати разломак:
\[
\variant{
\dfrac{6x^2y}{9xy}
}{
\dfrac{15a^3}{20a}
}{
\dfrac{14m^4n}{21m^2n}
}.
\]
\begin{solution}[\stretch 3]
\variant{
$\dfrac{2x}{3}$
}{
$\dfrac{3a^2}{4}$
}{
$\dfrac{2m^2}{3}$
}.
\end{solution}

\ifprintanswers\else\newpage\fi

\question[4]
Изрази као производ:
\[
\variant{
x^2-4x+4
}{
a^2+8a+16
}{
m^2-10m+25
}.
\]
\begin{solution}[\stretch 2]
\variant{
$(x-2)^2$
}{
$(a+4)^2$
}{
$(m-5)^2$
}.
\end{solution}

\question[4]
Помножи полиноме:
\[
\variant{
(x+5)(x-5)
}{
(2a+3)(a+4)
}{
(m-7)(m+2)
}.
\]
\begin{solution}[\stretch 3]
\variant{
$x^2-25$
}{
$2a^2+11a+12$
}{
$m^2-5m-14$
}.
\end{solution}

% =========================
% II. КРУГ
% =========================
\section*{Круг}

\question[3]
Полупречник круга је $r=\measure{\variant{3}{5}{7}}{cm}$. Израчунај пречник, обим и површину круга ($\pi\approx 3{,}14$).
\begin{solution}[\stretch 4]
Пречник: $d=2r=\measure{\variant{6}{10}{14}}{cm}$. \\
Обим: $O=2\pi r\approx\measure{\variant{18{,}84}{31{,}40}{43{,}96}}{cm}$. \\
Површина: $P=\pi r^2\approx\measure{\variant{28{,}26}{78{,}50}{153{,}86}}{cm^2}$.
\end{solution}

\ifprintanswers\else\newpage\fi

\question[4]
Дужина кружног лука је $l=\measure{\variant{15{,}7}{18{,}84}{31{,}4}}{cm}$ и одговара централном углу $\varphi=\variant{90^\circ}{120^\circ}{180^\circ}$. Израчунај полупречник круга ($\pi\approx 3{,}14$).
\begin{solution}[\stretch 3]
$l=\dfrac{\varphi}{360^\circ}\cdot 2\pi r$  
$r=\dfrac{l\cdot 360^\circ}{2\pi\varphi}$.  
\variant{
$r\approx\dfrac{15{,}7\cdot 360}{2\cdot 3{,}14\cdot 90}=\measure{10}{cm}$
}{
$r\approx\dfrac{18{,}84\cdot 360}{2\cdot 3{,}14\cdot 120}=\measure{9}{cm}$
}{
$r\approx\dfrac{31{,}4\cdot 360}{2\cdot 3{,}14\cdot 180}=\measure{10}{cm}$
}.
\end{solution}

\question[4]
Површина кружног сектора је $P_s=\measure{\variant{12{,}57}{25{,}13}{50{,}27}}{cm^2}$ и одговара централном углу $\varphi=\variant{90^\circ}{180^\circ}{270^\circ}$. Израчунај полупречник круга ($\pi\approx 3{,}14$).
\begin{solution}[\stretch 3]
$P_s=\dfrac{\varphi}{360^\circ}\cdot \pi r^2$  
$r=\sqrt{\dfrac{P_s\cdot 360^\circ}{\pi\varphi}}$.  
\variant{
$r\approx\sqrt{\dfrac{12{,}57\cdot 360}{3{,}14\cdot 90}}=\measure{4}{cm}$
}{
$r\approx\sqrt{\dfrac{25{,}13\cdot 360}{3{,}14\cdot 180}}=\measure{4}{cm}$
}{
$r\approx\sqrt{\dfrac{50{,}27\cdot 360}{3{,}14\cdot 270}}=\measure{4}{cm}$
}.
\end{solution}

\end{questions}

\end{document}
