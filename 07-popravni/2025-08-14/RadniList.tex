\documentclass[11pt,a5paper,twoside,addpoints,answers]{exam} % answers|noanswers

\usepackage[OT2]{fontenc}
\usepackage[utf8x]{inputenc}
\usepackage[serbian]{babel}
\usepackage{multicol}
\usepackage{amssymb,amsmath}
\usepackage{graphicx}
\usepackage{geometry}
\geometry{a5paper, margin=1.5cm}

% Макро за јединице мере
\newcommand{\measure}[2]{#1\,\mathrm{#2}}

% Макро за 3 варијанте
\newcommand{\variant}[3]{#1}

% ---------------- exam подешавања ----------------
\renewcommand{\solutiontitle}{\noindent\textrm{Решење:}\enspace}
\pointsinmargin
\pointname{}
\hqword{Задатак:}
\hpgword{Страница:}
\hpword{Поени:}
\hsword{Остварено:}
\htword{Збир}
\vqword{Задатак:}
\vpgword{Страница:}
\vpword{Поени:}
\vsword{Остварено:}
\vtword{Збир}
\cellwidth{1em}
\cfoot[]{Страница \thepage\ од \numpages}
\addto{\captionsserbian}{\renewcommand{\abstractname}{Упут{}ство}}

\title{Припремна настава}
\author{$\mathrm{VII}_\Box$ Полиноми 2/2. Круг
  \thanks{29 одлично,\; 22 врло добро,\; 15 добро,\; 8 довољно.}
}
\date{Станишић, 14.\ август 2025.}

\pagestyle{headandfoot}
\runningheader{Полиноми 2/2. Круг}{}{варијанта \variant{1}{2}{3}}
\runningfooter{}{}{Страна \thepage\ од \numpages}

\begin{document}
\maketitle
\thispagestyle{headandfoot}

\ifprintanswers\else
\begin{flushleft}\scriptsize
\gradetable[h]
\end{flushleft}
\fi

\begin{abstract}
Пажљиво читај текст задатка. Наводи поступак (кораке) и своди полиноме на сређен облик.
Код круга у већини задатака тражи се тачна вредност (са $\pi$); само ако је у загради дато $\pi\approx\cdots$, рачуна се приближно.
\end{abstract}

\begin{questions}

% =========================================================
% I. ПОЛИНОМИ — део 2 (квадрат бинома, разлика квадрата, растављање)
% =========================================================
\section*{Полиноми — део 2}

\question[3]
Примени формулу за квадрат бинома и среди израз:
\[
\variant{(3x-2)^2}{(2a+5)^2}{(4y-1)^2}.
\]
\begin{solution}[\stretch 2]
Формула: $(u\pm v)^2=u^2\pm 2uv+v^2$.
\[
\variant{
(3x)^2-2\cdot 3x\cdot 2+2^2=9x^2-12x+4
}{
(2a)^2+2\cdot 2a\cdot 5+5^2=4a^2+20a+25
}{
(4y)^2-2\cdot 4y\cdot 1+1^2=16y^2-8y+1}.
\]
\end{solution}

\question[2]
Примени разлику квадрата:
\[
\variant{(p-7)(p+7)}{(6x-5)(6x+5)}{(m-3n)(m+3n)}.
\]
\begin{solution}[\stretch 1]
$a^2-b^2=(a-b)(a+b)$, па је
\[
\variant{p^2-49}{36x^2-25}{m^2-9n^2}.
\]
\end{solution}

\question[3]
Растави на чиниоце:
\[
\variant{9x^2-12x+4}{25a^2-1}{8y^2+12y}.
\]
\begin{solution}[\stretch 2]
\variant{
$9x^2-12x+4=(3x-2)^2$
}{
$25a^2-1=(5a-1)(5a+1)$
}{
$8y^2+12y=4y(2y+3)$.
}
\end{solution}

\question[4]
Израчунај и среди:
\[
\variant{
(2x-3)(x+5)-x(x-4)
}{
(3a+2)(a-1)+ (a-4)(a+4)
}{
(5y-1)(y-2)- (2y-3)^2
}.
\]
\begin{solution}[\stretch 3]
\variant{
$2x^2+10x-3x-15-(x^2-4x)=
x^2+11x-15$
}{
$3a^2-3a+2a-2 + (a^2-16)=4a^2-a-18$
}{
$5y^2-10y -y+2 -\big(4y^2-12y+9\big)
= y^2+1y-7$.
}
\end{solution}

\question[4]
Растави на чиниоце применом дистрибутивности и формула:
\[
\variant{
6x^2-24= \;?
}{
12a^2+18a= \;?
}{
9m^2-4n^2= \;?
}
\]
\begin{solution}[\stretch 2]
\variant{
$6(x^2-4)=6(x-2)(x+2)$
}{
$6a(2a+3)$
}{
$(3m-2n)(3m+2n)$.
}
\end{solution}

\ifprintanswers\else\newpage\fi

% =========================
% II. КРУГ
% =========================
\section*{Круг}

\question[2]
Централни угао над луком износи
\(\variant{120^\circ}{150^\circ}{90^\circ}\).
Одреди периферијски угао над истим луком.
\begin{solution}[\stretch 1]
Периферијски угао је половина централног: 
\(\alpha=\tfrac{1}{2}\beta=
\variant{60^\circ}{75^\circ}{45^\circ}\).
\end{solution}

\question[3]
Круг има пречник
\(\variant{\measure{12}{cm}}{\measure{9}{cm}}{\measure{14}{cm}}\).
Израчунај \emph{тачно} обим и површину круга.
\begin{solution}[\stretch 2]
$O=d\pi$, $r=\frac d2$, $P=r^2\pi$.
\[
\variant{
O=12\pi,\ r=6,\ P=36\pi
}{
O=9\pi,\ r=4{,}5,\ P=\frac{81}{4}\pi
}{
O=14\pi,\ r=7,\ P=49\pi
}\ \ (\mathrm{cm},\ \mathrm{cm}^2).
\]
\end{solution}

\question[3]
Полупречник је
\(\variant{\measure{10}{cm}}{\measure{8}{cm}}{\measure{6}{cm}}\).
Колика је дужина лука за угао
\(\variant{72^\circ}{135^\circ}{30^\circ}\)?
\begin{solution}[\stretch 2]
$l=\dfrac{\alpha}{360^\circ}\cdot 2\pi r$.
\[
\variant{
l=\frac{72}{360}\cdot 2\pi\cdot 10
=4\pi\ \mathrm{cm}
}{
l=\frac{135}{360}\cdot 2\pi\cdot 8
=6\pi\ \mathrm{cm}
}{
l=\frac{30}{360}\cdot 2\pi\cdot 6
=\pi\ \mathrm{cm}
}.
\]
\end{solution}

\question[3]
Израчунај површину \emph{кружног исечка} са полупречником
\(\variant{\measure{7}{cm}}{\measure{9}{cm}}{\measure{5}{cm}}\)
и углом
\(\variant{120^\circ}{45^\circ}{150^\circ}\).
\begin{solution}[\stretch 2]
$P_{is}=\dfrac{\alpha}{360^\circ}\cdot \pi r^2$.
\[
\variant{
P_{is}=\frac{120}{360}\cdot \pi\cdot 7^2
=\frac{49}{3}\pi
}{
P_{is}=\frac{45}{360}\cdot \pi\cdot 9^2
=\frac{81}{8}\pi
}{
P_{is}=\frac{150}{360}\cdot \pi\cdot 5^2
=\frac{125}{12}\pi
}\ \mathrm{cm}^2.
\]
\end{solution}

\question[3]
У кружници полупречника
\(\variant{\measure{13}{cm}}{\measure{10}{cm}}{\measure{12}{cm}}\)
нацртана је тетива $AB$ чије је средиште на растојању
\(\variant{\measure{5}{cm}}{\measure{6}{cm}}{\measure{9}{cm}}\)
од центра. Израчунај дужину тетиве $AB$.
\begin{solution}[\stretch 2]
Полутетива $x$ и растојање $d$ чине катете, $r$ хипотенузу:
$x=\sqrt{r^2-d^2}$, $AB=2x$.
\[
\variant{
x=\sqrt{13^2-5^2}=12\Rightarrow AB=24\ \mathrm{cm}
}{
x=\sqrt{10^2-6^2}=8\Rightarrow AB=16\ \mathrm{cm}
}{
x=\sqrt{12^2-9^2}= \sqrt{63}=3\sqrt{7}\Rightarrow AB=6\sqrt{7}\ \mathrm{cm}
}.
\]
\end{solution}

\ifprintanswers\else\newpage\fi

\question[3]
\variant{
\textbf{(Конструкција)} Тангента на кружницу $k(O,r)$ у задатој тачки $T$ на кружници.
}{
\textbf{(Конструкција)} Дата је кружница $k(O,r)$ и тачка $M$ у унутрашњости.
Конструиши тетиву којој је $M$ средиште.
}{
\textbf{(Конструкција)} Дата је кружница $k(O,r)$ и спољашња тачка $P$.
Конструиши тангенте из $P$ на $k$.
}
\begin{solution}[\stretch 6]
\variant{
\emph{Поступак:}
\begin{enumerate}\item Спој $O$ и $T$ (полупречник $OT$).
\item Конструиши праву кроз $T$ нормалну на $OT$ (нпр. помоћу симетрале угла или једнаког полукруга).
\item Та права је тражена тангента у $T$.
\end{enumerate}
\emph{Образложење:} Полупречник је нормалан на тангенту у додирној тачки.
}{
\emph{Поступак:}
\begin{enumerate}\item Нацртај праву $OM$.
\item На прави кроз $M$ конструиши нормалу на $OM$.
\item Пресеци те нормале са кружницом су крајеви тражене тетиве $AB$.
\end{enumerate}
\emph{Образложење:} Средиште тетиве лежи на нормали на тетиву кроз центар; дакле $OM\perp AB$ и $M$ је средина $AB$.
}{
\emph{Поступак (тангента из спољашње тачке):}
\begin{enumerate}
\item Спој $O$ и $P$.
\item Конструиши симетралу дужи $OP$; нека је $S$ њено средиште.
\item Описана кружница $k_S(S,SO)$ сече $k$ у додирним тачкама $T_1,T_2$.
\item Права $PT_1$ и $PT_2$ су тражене тангенте.
\end{enumerate}
\emph{Образложење:} $ST_1=SO=SP$, па је $\angle OT_1P$ прав (Талесова теорема), те $PT_1\perp OT_1$.
}
\end{solution}

\question[4]
\variant{
У квадрат страница $\measure{8}{cm}$ уписана је кружница, а затим око квадрата описана кружница.
Израчунај тачну површину \emph{кружног прстена} између описане и уписане кружнице.
}{
У једнакостранични троугао страница $\measure{6}{cm}$ уписана је кружница, а затим описана кружница.
Израчунај тачну површину \emph{кружног прстена} између њих.
}{
Правоугаоник има странице $\measure{6}{cm}$ и $\measure{8}{cm}$.
Уписана и описана кружница не постоје обе; зато израчунај \emph{тачну} површину круга чији је пречник једнак дужини дијагонале тог правоугаоника.
}
\begin{solution}[\stretch 4]
\variant{
За квадрат $a$, $r_u=\tfrac a2=4$, $r_o=\tfrac{a\sqrt2}{2}=4\sqrt2$.
$P=\pi(r_o^2-r_u^2)=\pi\,(32-16)=16\pi\ \mathrm{cm}^2$.
}{
За једнакостранични троугао $a$, $r_u=\tfrac{a\sqrt3}{6}$, $r_o=\tfrac{a\sqrt3}{3}$.
За $a=6$: $r_u=\sqrt3$, $r_o=2\sqrt3$.
$P=\pi( (2\sqrt3)^2-(\sqrt3)^2 )=\pi(12-3)=9\pi\ \mathrm{cm}^2$.
}{
Дијагонала: $d=\sqrt{6^2+8^2}=10$ (Питагорина).
Круг са пречником $d$ има полупречник $5$, па је $P=25\pi\ \mathrm{cm}^2$.
}
\end{solution}

\ifprintanswers\else\newpage\fi

\question[3]
\variant{
Тачка $P$ је на растојању $\measure{13}{cm}$ од центра круга полупречника $\measure{5}{cm}$.
Колика је дужина тангентне дужи из $P$ на круг?
}{
Полупречник круга је $\measure{10}{cm}$. Израчунај дужину лука од $\,\measure{25}{cm}$ у степенима (централни угао).
}{
Полупречник круга је $\measure{6}{cm}$. Нађи површину круга \emph{приближно} ($\pi\approx3{,}14$).
}
\begin{solution}[\stretch 2]
\variant{
У $\triangle OPT$ је $OP^2=OT^2+PT^2$, $OT=r=5$.
$PT=\sqrt{13^2-5^2}=\sqrt{169-25}=12\ \mathrm{cm}$.
}{
$l=\dfrac{\alpha}{360^\circ}\cdot 2\pi r\Rightarrow
\alpha=\dfrac{360^\circ\,l}{2\pi r}
=\dfrac{360\cdot 25}{20\pi}
=\dfrac{450}{\pi}^\circ\approx 143{,}24^\circ.$
}{
$P=r^2\pi=36\cdot 3{,}14=113{,}04\ \mathrm{cm}^2$.
}
\end{solution}

\end{questions}

\end{document}
