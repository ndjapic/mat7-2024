\documentclass[11pt,a5paper,twoside,addpoints,noanswers]{exam}

\usepackage[OT2]{fontenc}
\usepackage[utf8x]{inputenc}
\usepackage[serbian]{babel}
\usepackage{amsmath,amssymb,geometry}
\geometry{a5paper, margin=1.5cm}

% Јединице мере
\newcommand{\measure}[2]{#1\,\mathrm{#2}}

% Макро за 3 варијанте
\newcommand{\variant}[1]{#1}

\renewcommand{\solutiontitle}{\noindent\textbf{Решење:}\par\noindent}
\addto{\captionsserbian}{\renewcommand{\abstractname}{Упут{}ство}}

\pointsinmargin
\pointname{}

\title{Писмени део поправног испита}
\author{Математика, 7. разред}
\date{Трајање: 45 минута}

\begin{document}
\maketitle

\ifprintanswers\else
\begin{flushleft}\scriptsize
\gradetable[h]
\end{flushleft}
\begin{abstract}\small
Пажљиво читај текст задатка. Наводи поступак (кораке) и своди полиноме на сређен облик.  
Код круга се тражи \emph{тачна} вредност (са $\pi$), осим ако у задатку пише приближна вредност $\pi$.
\end{abstract}
\fi

\ifprintanswers\else
\newpage
\fi

% =====================================================
% Варијанта 1
% =====================================================
\variant{Варијанта 1}

\begin{questions}

\question[4] Растави на чиниоце: $x^2+6x+9$.
\begin{solution}[\stretch{2}]
$x^2+6x+9=(x+3)(x+3)=(x+3)^2$.  
Поступак:  
1. Препознајемо да је израз облик $a^2+2ab+b^2$.  
2. Упоређујемо: $x^2+2\cdot 3x+3^2$.  
3. Дакле, резултат је $(x+3)^2$.
\end{solution}

\question[3] Израчунај површину круга полупречника $r=7$.
\begin{solution}[\stretch{2}]
$P=\pi r^2=\pi\cdot 7^2=49\pi$.  
Кораци:  
1. Запишемо формулу $P=\pi r^2$.  
2. Уместо $r$ упишемо $7$.  
3. Добијемо $49\pi$.
\end{solution}

\question[4] Израчунај дужину тетиве у кругу полупречника $13$ ако је растојање од центра до тетиве $5$.
\begin{solution}[\stretch{3}]
Правимо правоугли троугао: полупречник, полутетива, растојање.  
Питагорина теорема: $13^2=5^2+(\tfrac{c}{2})^2$.  
$(\tfrac{c}{2})^2=169-25=144$, $\tfrac{c}{2}=12$.  
$c=24$.  
\end{solution}

\question[3] Израчунај број дијагонала у седмоуглу.
\begin{solution}[\stretch{2}]
Формула: $D=\tfrac{n(n-3)}2$.  
За $n=7$: $D=\tfrac{7\cdot 4}2=14$.  
\end{solution}

\question[4] Конструиши тангенту на кружницу $k(O,r)$ у тачки $T$.
\begin{solution}[\stretch{4}]
1. Спојити $O$ и $T$.  
2. Конструисати нормалу на $OT$ у тачки $T$ (користећи конструисање правог угла).  
3. Добијена права је тражена тангента.  
\end{solution}

\question[4] Израчунај површину кружног исечка ако је $r=6$, $\alpha=120^\circ$.
\begin{solution}[\stretch{3}]
$P=\frac{\alpha}{360^\circ}\pi r^2=\frac{120}{360}\pi\cdot 36=12\pi$.  
\end{solution}

\end{questions}

\end{document}
