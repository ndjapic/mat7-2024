\documentclass[11pt,a5paper,twoside,addpoints,noanswers]{exam}

\usepackage[OT2]{fontenc}
\usepackage[utf8x]{inputenc}
\usepackage[serbian]{babel}
\usepackage{amsmath,amssymb,geometry}
\geometry{a5paper, margin=1.5cm}

% Јединице мере
\newcommand{\measure}[2]{#1\,\mathrm{#2}}

% Макро за 3 варијанте
\newcommand{\variant}[1]{#1}

\renewcommand{\solutiontitle}{\noindent\textbf{Решење:}\par\noindent}
\addto{\captionsserbian}{\renewcommand{\abstractname}{Упут{}ство}}

\pointsinmargin
\pointname{}

\title{Писмени део поправног испита}
\author{Математика, 7. разред}
\date{Трајање: 45 минута}

\begin{document}
\maketitle

\ifprintanswers\else
\begin{flushleft}\scriptsize
\gradetable[h]
\end{flushleft}
\begin{abstract}\small
Пажљиво читај текст задатка. Наводи поступак (кораке) и своди полиноме на сређен облик.  
Код круга се тражи \emph{тачна} вредност (са $\pi$), осим ако у задатку пише приближна вредност $\pi$.
\end{abstract}
\fi

\ifprintanswers\else
\newpage
\fi

% =====================================================
% Варијанта 3
% =====================================================
\variant{Варијанта 3}

\begin{questions}

\question[4] Израчунај површину круга пречника $10$.
\begin{solution}[\stretch{2}]
$r=5$. $P=\pi r^2=\pi\cdot 25=25\pi$.  
\end{solution}

\question[3] Растави на чиниоце: $x^2+2x$.
\begin{solution}[\stretch{2}]
$x^2+2x=x(x+2)$.  
Поступак: износимо заједнички чинилац $x$.  
\end{solution}

\question[4] Израчунај дужину тетиве у кругу $r=10$ ако је угао при центру $60^\circ$.
\begin{solution}[\stretch{3}]
У једнакостраничном троуглу $OAB$: $OA=OB=10$, $\angle AOB=60^\circ$.  
Дакле $AB=10$.  
\end{solution}

\question[3] Израчунај број дијагонала у деветоуглу.
\begin{solution}[\stretch{2}]
$D=\tfrac{9\cdot 6}2=27$.  
\end{solution}

\question[4] Израчунај површину кружног прстена између кругова $R=5$, $r=3$.
\begin{solution}[\stretch{3}]
$P=\pi R^2-\pi r^2=\pi(25-9)=16\pi$.  
\end{solution}

\question[4] Конструиши тангенте на кружницу $k(O,r)$ из спољашње тачке $A$.
\begin{solution}[\stretch{4}]
1. Повезати $O$ и $A$.  
2. На $OA$ конструисати полупречник $OT$ тако да је $\angle OAT$ прав.  
3. Кроз $A$ повући тангенте.  
\end{solution}

\end{questions}

\end{document}
