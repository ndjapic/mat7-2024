\documentclass[11pt,a5paper,twoside,addpoints,noanswers]{exam}

\usepackage[OT2]{fontenc}
\usepackage[utf8x]{inputenc}
\usepackage[serbian]{babel}
\usepackage{amsmath,amssymb,geometry}
\geometry{a5paper, margin=1.5cm}

% Јединице мере
\newcommand{\measure}[2]{#1\,\mathrm{#2}}

% Макро за 3 варијанте
\newcommand{\variant}[1]{#1}

\renewcommand{\solutiontitle}{\noindent\textbf{Решење:}\par\noindent}
\addto{\captionsserbian}{\renewcommand{\abstractname}{Упут{}ство}}

\pointsinmargin
\pointname{}

\title{Писмени део поправног испита}
\author{Математика, 7. разред}
\date{Трајање: 45 минута}

\begin{document}
\maketitle

\ifprintanswers\else
\begin{flushleft}\scriptsize
\gradetable[h]
\end{flushleft}
\begin{abstract}\small
Пажљиво читај текст задатка. Наводи поступак (кораке) и своди полиноме на сређен облик.  
Код круга се тражи \emph{тачна} вредност (са $\pi$), осим ако у задатку пише приближна вредност $\pi$.
\end{abstract}
\fi

\ifprintanswers\else
\newpage
\fi

% =====================================================
% Варијанта 2
% =====================================================
\variant{Варијанта 2}

\begin{questions}

\question[3] Израчунај обим круга ако је $d=14$.
\begin{solution}[\stretch{2}]
$O=\pi d=\pi\cdot 14=14\pi$.  
\end{solution}

\question[4] Растави на чиниоце: $x^2-25$.
\begin{solution}[\stretch{2}]
$x^2-25=(x-5)(x+5)$.  
Поступак: користимо формулу $a^2-b^2=(a-b)(a+b)$.  
\end{solution}

\question[4] Израчунај дужину лука круга ако је $r=10$, угао $45^\circ$.
\begin{solution}[\stretch{3}]
$l=\frac{\alpha}{360^\circ}\cdot 2\pi r=\frac{45}{360}\cdot 20\pi=\tfrac{1}{8}\cdot 20\pi=2.5\pi$.  
\end{solution}

\question[4] Колико дијагонала има десетоугао?
\begin{solution}[\stretch{2}]
$D=\tfrac{10\cdot 7}{2}=35$.  
\end{solution}

\question[3] Израчунај обим кружног исечка $r=12$, $\alpha=90^\circ$.
\begin{solution}[\stretch{3}]
Обим исечка: $O=2r+l=24+ \frac{90}{360}\cdot 2\pi\cdot 12=24+6\pi$.  
\end{solution}

\question[4] Конструиши кружницу уписану у ромб.
\begin{solution}[\stretch{4}]
1. Конструисати ромб.  
2. Повезати симетрале углова.  
3. Њихов пресек је центар уписане кружнице.  
4. Описати кружницу са тим центром и полупречником до странице.  
\end{solution}

\end{questions}

\end{document}
