\documentclass[10pt,a5paper,addpoints]{exam}

\usepackage[a5paper,margin=1.5cm]{geometry}
\usepackage{multicol}
\usepackage{amssymb,amsmath}
\usepackage{graphicx}
\usepackage[OT2]{fontenc}
\usepackage[utf8x]{inputenc}
\usepackage[serbian]{babel}

% Макро за мере (користити као: \measure{\variant{3}{4}{5}}{cm})
\newcommand{\measure}[2]{#1\,\mathrm{#2}}

% Макро за варијанте (3 варијанте)
% Промените дефиницију на \def\variant#1#2#3{#2} или {#3} за варијанту 2 или 3
\def\variant#1#2#3{#1}

% Ако желите да штампате решења, откоментаришите следећи ред:
%\printanswers

\renewcommand{\solutiontitle}{\noindent\textbf{Решење:}\enspace}
\pointsinmargin
\pointname{}
\hqword{Задатак:}
\hpgword{Страница:}
\hpword{Поени:}
\hsword{Остварено:}
\htword{Збир}
\cellwidth{0.9em}
\cfoot[]{Страна \thepage\ од \numpages}

\title{Припремна настава – Дан 3\\ \small Многоугао \& Полиноми (1/2)}
\author{$\mathrm{VII}_\Box$, варијанта \variant{1}{2}{3}}
\date{Станишић, август 2025.}

\begin{document}
\maketitle
\thispagestyle{headandfoot}

\noindent Име и презиме:\enspace\underline{\makebox[0.45\textwidth][s]{\hfill}}

\begin{flushleft}
\gradetable[h]
\end{flushleft}

\begin{abstract}
Пре израде пажљиво прочитај задатке. Решења се налазе одмах испод задатака (остављено да наставник открије по потреби). Трајање: око 40 минута (за припрему и вежбу — делове задатака урадити и код куће).
\end{abstract}

\begin{questions}

% -------------------------
% МНОГОУГАO
% -------------------------
\section*{I Многоугао}

\question[3]
Одреди број страница и збир унутрашњих углова правилног \variant{шестоугла}{осмоугла}{десетоугла}:
\begin{parts}
\part Колико страница има \(n\)?
\part Израчунај збир унутрашњих углова.
\end{parts}

\begin{solution}
\begin{parts}
\part Правилан \( \) \variant{шестоугао}{осмоугао}{десетоугао} има $n=\variant{6}{8}{10}$ страна.
\part Збир унутрашњих углова за $n$-угaо је $S=(n-2)\cdot 180^\circ$. Дакле
\[
S=(\variant{6}{8}{10}-2)\cdot 180^\circ = \variant{4\cdot 180^\circ}{6\cdot 180^\circ}{8\cdot 180^\circ}
= \variant{720^\circ}{1080^\circ}{1440^\circ}.
\]
\end{parts}
\end{solution}

\question[3]
Правилан \variant{шестоугао}{осмоугао}{десетоугао} има страницу \(a=\measure{\variant{6}{4}{3}}{cm}\). Израчунај обим и приближну површину (користи формулу за површину правилног n-угла: \(P = \dfrac{n a^2}{4\tan(\pi/n)}\) или познатије облике за шестоугао/осмоугао).
\begin{solution}
Обим: $\mathcal{O} = n a = \variant{6\cdot 6}{8\cdot 4}{10\cdot 3}\,\mathrm{cm}
= \measure{\variant{36}{32}{30}}{cm}.$

Површина: користимо општу формулу
\[
P=\frac{n a^2}{4\tan(\pi/n)}.
\]
За варијанту 1 (шестоугао): $P = \frac{6\cdot 6^2}{4\tan(\pi/6)} = \frac{216}{4\cdot \tfrac{1}{\sqrt3}} = 54\sqrt3 \,\mathrm{cm}^2$.  
За варијанту 2 (осмоугао) и 3 (десетоугао) примењује се иста формула са одговарајућом вредношћу $\tan(\pi/n)$; записати приближно или оставити у формулском облику по потреби.
\end{solution}

\question[3]
Колико дијагонала има \variant{дванаестоугао}{шеснаестоугао}{четвороугао} и колико износи збир унутрашњих углова тог многоугла?
\begin{solution}
Број дијагонала: $D=\dfrac{n(n-3)}2$.

За варијанту 1: $n=12$,
$D=\dfrac{12\cdot9}{2}=54$. Збир углова: $S=(12-2)\cdot180^\circ=1800^\circ$.

За варијанту 2: $n=16$,
$D=\dfrac{16\cdot13}{2}=104$. Збир: $(16-2)\cdot180^\circ=2520^\circ$.

За варијанту 3: $n=4$,
$D=\dfrac{4\cdot1}{2}=2$ (то су дијагонале квадрата). Збир: $(4-2)\cdot180^\circ=360^\circ$.
\end{solution}

\question[4]
Конструиши троугао $ABC$ дат следећим елементима: страну $AB=\measure{\variant{6}{5}{7}}{cm}$ и углове у теменима $A$ и $B$ једнаке \variant{50^\circ и 60^\circ}{40^\circ и 50^\circ}{55^\circ и 45^\circ} (подаци у варијантама). Напиши кораке конструкције и нацртај резултат (користећи лењир и шестар).
\begin{solution}
Кораци конструкције (ASA): \\
1. Нацртај дуж $AB=\measure{\variant{6}{5}{7}}{cm}$. \\
2. У тачки $A$ подеси угао $ \variant{50^\circ}{40^\circ}{55^\circ}$ помоћу шестара/менталног мерења и нацртај полуправу која прави тај угао са $AB$. \\
3. У тачки $B$ подеси угао $ \variant{60^\circ}{50^\circ}{45^\circ}$ и нацртај полуправу која прави тај угао са $BA$. \\
4. Пресек две полуправе је тачка $C$. Повежи $C$ са $A$ и $B$. \\
Образложење: Пресек полуправи постоји јер су углови такви да су полуправи нехомолинеарне; добијени троугао је једнозначан по ASA. (Ученик треба да приложи цртеж.)
\end{solution}

\question[4]
(Доказ/вежба) Докажи да у ромбу дијагонале деле унутрашње углове (краћи доказ, нацртати и навести подударања троуглова).
\begin{solution}
Нека је $ABCD$ ромб. Дијагонале су $AC$ и $BD$, нека се секу у $O$. Посматрамо троуглове $\triangle AOB$ и $\triangle AOD$. Имамо $AO$ заједничко; $AB=AD$ (странице ромба); $BO=DO$ (јер дијагонале ромба једнако деле). Можемо применити подударност (ССС или УСУ у зависности од аргумената) и закључити да су углови наспрам једнаких страница једнаки, дакле дијагонала $AC$ дели унутрашњи угао у $A$ на два једнака дела. (Кратак цртеж и подударање довољни.)
\end{solution}

% -------------------------
% ПОЛИНОМИ – ДЕО 1
% -------------------------
\section*{II Полиноми — део 1 (мономи и операције)}

\question[2]
Одреди степен и коефицијент монома $\variant{5x^3y^2}{-4a^2b}{\tfrac{1}{3}c}$.
\begin{solution}
Варијанта 1: $5x^3y^2$ — коефицијент $5$, степен $3+2=5$.\\
Варијанта 2: $-4a^2b$ — коефицијент $-4$, степен $2+1=3$.\\
Варијанта 3: $\tfrac{1}{3}c$ — коефицијент $\tfrac{1}{3}$, степен $1$.
\end{solution}

\question[2]
Израчунај вредност полинома \(P(x)=\variant{3x^2-5x+2}{-2x^3+4x-1}{2x^2-3x+1}\) за \(x=\variant{-1}{2}{-1}\).
\begin{solution}
Варијанта 1: $P(x)=3x^2-5x+2$, за $x=-1$: $3\cdot1 -5(-1)+2 = 3+5+2=10$.\\
Варијанта 2: $P(x)=-2x^3+4x-1$, за $x=2$: $-2\cdot8 + 4\cdot2 -1 = -16+8-1=-9$.\\
Варијанта 3: $P(x)=2x^2-3x+1$, за $x=-1$: $2\cdot1 -3(-1)+1=2+3+1=6$.
\end{solution}

\question[3]
Сведи на сређен облик и уреди:
\[
(4a^2 - 3a + 1) + (a^2 + 5a - 2)
\]
(варијанта 1),
\[
(5x^2 + 2x -3) - (2x^2 - x +4)
\]
(варијанта 2),
\[
(3x^2 - x) + (2x^2 + 4x -1)
\]
(варијанта 3).
\begin{solution}
В1: $4a^2+a^2 -3a+5a +1-2 = 5a^2 +2a -1$.\\
В2: $5x^2-2x^2 +2x -(-x) -3-4 = 3x^2 +3x -7$. (Проверити знаке.)\\
В3: $3x^2+2x^2 -x +4x -1 = 5x^2 +3x -1$.
\end{solution}

\question[3]
Помножи мономе:
\[
(-2x^3y)\cdot (5xy^2)
\]
(варијанта 1),
\[
(3ab^2c^2)\cdot (-4a^2bc^3)
\]
(варијанта 2),
\[
(3a^2b)\cdot ( -5ab^2 )
\]
(варијанта 3).
\begin{solution}
В1: $(-2x^3y)\cdot(5xy^2) = -10 x^{4} y^{3}$.\\
В2: $(3ab^2c^2)\cdot(-4a^2bc^3) = -12 a^{3} b^{3} c^{5}$.\\
В3: $(3a^2b)\cdot(-5ab^2) = -15 a^{3} b^{3}$.
\end{solution}

\question[4]
Помножи полиноме (моном \(\times\) полином или полином \(\times\) полином):
\[
\text{варијанта 1: } (x-2)(x^2+2x+4),
\qquad
\text{варијанта 2: } (2x+1)(x^2-3x+2),
\qquad
\text{варијанта 3: } 4x^2(x^4+3x^3-2x-3).
\]
\begin{solution}
В1: $(x-2)(x^2+2x+4) = x^3 +2x^2 +4x -2x^2 -4x -8 = x^3 -8$.\\
В2: $(2x+1)(x^2-3x+2) = 2x^3 -6x^2 +4x + x^2 -3x +2 = 2x^3 -5x^2 + x +2$.\\
В3: $4x^2\cdot(x^4+3x^3-2x-3) = 4x^6 +12x^5 -8x^3 -12x^2$.
\end{solution}

\question[4]
Упрости израз (без коришћења формула за квадрат бинома):
\[
(2x-3)(x+4) - 2x(x-1).
\]
\begin{solution}
Прво проширимо: $(2x-3)(x+4)=2x^2+8x-3x-12=2x^2+5x-12$. \\
Други део: $2x(x-1)=2x^2-2x$. \\
Разлика: $(2x^2+5x-12)-(2x^2-2x)=7x-12$.
\end{solution}

\question[4]
(Комбиновани задатак) Израчунај и уреди:
\[
a(2a+3) + (3a+2)(a-1).
\]
\begin{solution}
Разложимо: $a(2a+3)=2a^2+3a$. \\
$(3a+2)(a-1)=3a^2-3a+2a-2=3a^2 - a -2$. \\
Збир: $2a^2+3a + 3a^2 - a -2 = 5a^2 +2a -2$.
\end{solution}

\question[3]
Нацртај (украдички) дрво израза за: \(\, (a+b)\cdot(a+b) - 3ab \,\). Објасни које операције се раде прво.
\begin{solution}
Дрво: корен — одузимање; лева грана — множење $(a+b)(a+b)$ (стандардно: прво сабирање у заградама, па множење); десна грана — множити $3\cdot ab$. Операција приоритета: израчунавају се садржаји заграда, па множења, па сабирања/одузимања.
\end{solution}

\question[3]
(Домаћи/вежба) Напиши пример монома и одреди за њега степен и коефицијент. (Уз кратко образложење.)
\begin{solution}
Пример: $-6x^2y^3$. Коефицијент је $-6$, степен је $2+3=5$. Ученик да наведе свој пример.
\end{solution}

\end{questions}

\end{document}
