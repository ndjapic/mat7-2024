\documentclass[10pt,a5paper,addpoints]{exam}

\usepackage[a5paper,margin=1.5cm]{geometry}
\usepackage{multicol}
\usepackage{amssymb,amsmath}
\usepackage{graphicx}
\usepackage[OT2]{fontenc}
\usepackage[utf8x]{inputenc}
\usepackage[serbian]{babel}

% Макро за јединице мере (користити у мат. режиму)
\newcommand{\measure}[2]{#1\,\mathrm{#2}}

% Макро за 3 варијанте (подесити који аргумент је активан)
\def\variant#1#2#3{#1}

% Ако желите да се прикажу решења, откоментаришите:
% \printanswers

\renewcommand{\solutiontitle}{\noindent\textrm{Решење:}\enspace}
\pointsinmargin
\pointname{}
\hqword{Задатак:}
\hpgword{Страница:}
\hpword{Поени:}
\hsword{Остварено:}
\htword{Збир}
\cellwidth{0.9em}
\cfoot[]{Страна \thepage\ од \numpages}

\title{Припремна настава – Дан 3\\ \small Многоугао \& Полиноми (1/2)}
\author{$\mathrm{VII}_\Box$, варијанта \variant{1}{2}{3}}
\date{Станишић, август 2025.}

\begin{document}
\maketitle
\thispagestyle{headandfoot}

\noindent Име и презиме:\enspace\underline{\makebox[0.45\textwidth][s]{\hfill}}

\begin{flushleft}
\gradetable[h]
\end{flushleft}

\begin{abstract}
Пре израде пажљиво прочитај текст задатка. Кратко и јасно наведи поступак. Решења полинома своди на сређен облик.
\end{abstract}

\begin{questions}

% =========================
% I. МНОГОУГАО
% =========================
\section*{I Многоугао}

\question[3]
Одреди број страница $n$ и збир унутрашњих углова $S$ правилног
\variant{шестоугла}{осмоугла}{десетоугла}.
\begin{solution}[\stretch 2]
$n=\variant{6}{8}{10}$, а $S=(n-2)\cdot 180^\circ
= (\variant{6}{8}{10}-2)\cdot 180^\circ
= \variant{720^\circ}{1080^\circ}{1440^\circ}$.
\end{solution}

\question[3]
Правилан \variant{шестоугао}{троугао}{шестоугао} има страницу
$\measure{\variant{6}{12}{4}}{cm}$.
Израчунај обим и површину.
\begin{solution}[\stretch 2]
Обим: $\mathcal{O}=n\cdot a=\variant{6}{3}{6}\cdot
\measure{\variant{6}{12}{4}}{cm}
= \measure{\variant{36}{36}{24}}{cm}$.

Површина: за једнакостранични троугао $P=\dfrac{\sqrt{3}}{4}a^2$,
за шестоугао $P=6\cdot\dfrac{\sqrt{3}}{4}a^2=\dfrac{3\sqrt{3}}{2}a^2$.
Дакле,
\[
P=\variant{
\dfrac{3\sqrt{3}}{2}\cdot 6^2
= 54\sqrt{3}\,\mathrm{cm}^2
}{
\dfrac{\sqrt{3}}{4}\cdot 12^2
= 36\sqrt{3}\,\mathrm{cm}^2
}{
\dfrac{3\sqrt{3}}{2}\cdot 4^2
= 24\sqrt{3}\,\mathrm{cm}^2
}.
\]
\end{solution}

\question[3]
Израчунај број дијагонала и збир унутрашњих углова конвексног
$n$-угла за \(n=\variant{12}{15}{10}\).
\begin{solution}[\stretch 2]
Број дијагонала: $D=\dfrac{n(n-3)}{2}
=\variant{\dfrac{12\cdot 9}{2}=54}{\dfrac{15\cdot 12}{2}=90}{\dfrac{10\cdot 7}{2}=35}$.
Збир: $S=(n-2)\cdot 180^\circ
=\variant{1800^\circ}{2340^\circ}{1440^\circ}$.
\end{solution}

\ifprintanswers\else\newpage\fi

\question[4]
(Конструкција, УСУ) Конструиши троугао $ABC$ ако је
$AB=\measure{\variant{6}{5}{7}}{cm}$,
$\measuredangle A=\variant{45^\circ}{30^\circ}{60^\circ}$ и
$\measuredangle B=\variant{60^\circ}{75^\circ}{45^\circ}$.
Напиши кораке конструкције.
\begin{solution}[\stretch 3]
1) Нацртај дуж $AB=\measure{\variant{6}{5}{7}}{cm}$. \\
2) У тачки $A$ конструиши угао $\variant{45^\circ}{30^\circ}{60^\circ}$ и повуци полуправу. \\
3) У тачки $B$ конструиши угао $\variant{60^\circ}{75^\circ}{45^\circ}$ и повуци полуправу. \\
4) Пресек полуправих је $C$. Повежи $C$ са $A$ и $B$. \\
Једнозначност по УСУ (страница и два налегла угла).
\end{solution}

\question[4]
(Доказ, подударност) Докажи да у ромбу дијагонала дели унутрашњи угао на два једнака дела.
\begin{solution}[\stretch 3]
Нека је $ABCD$ ромб и нека је дијагонала $AC$.
Посматрај $\triangle ABC$ и $\triangle ADC$.
Имамо $AB=AD$ и $BC=DC$ (странице ромба су једнаке) и заједничку страницу $AC$.
По ССС: $\triangle ABC \cong \triangle ADC$.
Па су $\measuredangle BAC=\measuredangle CAD$, тј. $AC$ дели угао код $A$ на два једнака дела.
\end{solution}

% =========================
% II. ПОЛИНОМИ – 1. део
% =========================
\section*{II Полиноми — део 1}

\question[2]
Одреди коефицијент и степен монома
$\variant{5x^3y^2}{-4a^2b}{\tfrac{1}{3}c}$.
\begin{solution}[\stretch 2]
Коеф. и степен су:
\variant{$5$, $3+2=5$}{\(-4\), \(2+1=3\)}{\(\tfrac{1}{3}\), \(1\)}.
\end{solution}

\question[2]
Израчунај вредност полинома
$P(x)=\variant{3x^2-5x+2}{-2x^3+4x-1}{2x^2-3x+1}$
за $x=\variant{-1}{2}{-1}$.
\begin{solution}[\stretch 2]
\variant{
$3(-1)^2-5(-1)+2=3+5+2=10$
}{
$-2\cdot 2^3+4\cdot 2-1=-16+8-1=-9$
}{
$2(-1)^2-3(-1)+1=2+3+1=6$
}.
\end{solution}

\ifprintanswers\else\newpage\fi

\question[3]
Среди полином:
\[
\variant{
(4a^2 - 3a + 1) + (a^2 + 5a - 2)
}{
(5x^2 + 2x - 3) - (2x^2 - x + 4)
}{
(3x^2 - x) + (2x^2 + 4x - 1)
}.
\]
\begin{solution}[\stretch 3]
\variant{
$5a^2+2a-1$
}{
$3x^2+3x-7$
}{
$5x^2+3x-1$
}.
\end{solution}

\question[3]
Помножи мономе:
\[
\variant{
(-2x^3y)\cdot (5xy^2)
}{
(3ab^2c^2)\cdot (-4a^2bc^3)
}{
(3a^2b)\cdot (-5ab^2)
}.
\]
\begin{solution}[\stretch 2]
\variant{
$-10x^4y^3$
}{
$-12a^3b^3c^5$
}{
$-15a^3b^3$
}.
\end{solution}

\question[4]
Израчунај производ:
\[
\variant{
(x-2)(x^2+2x+4)
}{
(2x+1)(x^2-3x+2)
}{
4x^2(x^4+3x^3-2x-3)
}.
\]
\begin{solution}[\stretch 3]
\variant{
$x^3-8$
}{
$2x^3-5x^2+x+2$
}{
$4x^6+12x^5-8x^3-12x^2$
}.
\end{solution}

\ifprintanswers\else\newpage\fi

\question[4]
Упрости:
\[
\variant{
(2x-3)(x+4) - 2x(x-1)
}{
(3a+2)(a-1) + a(2a+3)
}{
(x-2)(3x+1) - (x^2+2x)
}.
\]
\begin{solution}[\stretch 4]
\variant{
$(2x^2+5x-12)-(2x^2-2x)=7x-12$
}{
$(3a^2-a-2)+(2a^2+3a)=5a^2+2a-2$
}{
$(3x^2-5x-2)-(x^2+2x)=2x^2-7x-2$
}.
\end{solution}

\end{questions}

\end{document}
