%\documentclass[11pt,a5paper,twoside,addpoints,noanswers]{exam} % задаци
\documentclass[11pt,a5paper,twoside,addpoints,answers]{exam} % решења

\usepackage[OT2]{fontenc}
\usepackage[utf8x]{inputenc}
\usepackage[serbian]{babel}
\usepackage{multicol}
\usepackage{amssymb,amsmath}
\usepackage{graphicx}
\usepackage{geometry}
\geometry{a5paper, margin=1.5cm}

% Макро за јединице мере
\newcommand{\measure}[2]{\mathrm{#1\,#2}}

% Макро за варијанте (пример: \variant{a}{b}{c})
\newcommand{\variant}[3]{#1}

% Подешавања за exam
%\printanswers % <- уклони % да се виде решења
\renewcommand{\solutiontitle}{\noindent\textrm{Решење:}\enspace}
\pointsinmargin
\pointname{}
\hqword{Задатак:}
\hpgword{Страница:}
\hpword{Поени:}
\hsword{Остварено:}
\htword{Збир}
\cellwidth{1em}
\cfoot[]{Страница \thepage\ од \numpages}
\addto{\captionsserbian}{\renewcommand{\abstractname}{Упут{}ство}}

\title{$\mathrm{VII}$ разред, варијанта \variant{1}{2}{3}}
\author{Припремна настава – Први дан}
\date{Станишић, 11.~август 2025.}

\begin{document}
\maketitle
\thispagestyle{headandfoot}

\begin{center}
\gradetable[h]
\end{center}

\begin{questions}

\section{Лакши задаци (загревање)}

\question[2]
Израчунај:
\begin{multicols}{3}
\begin{parts}
\part $\sqrt{\variant{49}{81}{121}}$
\ifprintanswers
\else\answerline
\fi
%\part $\sqrt[3]{\variant{125}{64}{216}}$ % Кубни корен није у плану у основној школи.
\part $\left|\variant{-5}{8}{-12}\right|$ % Ученик треба да разликује апсолутну од супротне вредности.
\ifprintanswers
\else\answerline
\fi
\part $\frac{\variant{24}{36}{48}}{\variant{6}{9}{16}}$
\ifprintanswers
\else\answerline
\fi
\end{parts}
\end{multicols}
\begin{solution}[\stretch 1]
\begin{parts}
\begin{multicols}{3}
\part $\sqrt{\variant{49}{81}{121}} = \variant{7}{9}{11}$
%\part $\sqrt[3]{\variant{125}{64}{216}} = \variant{5}{4}{6}$
\part $|\variant{-5}{8}{-12}| = \variant{5}{8}{12}$
\part $\frac{\variant{24}{36}{48}}{\variant{6}{9}{16}} = \variant{4}{4}{3}$
\end{multicols}
\end{parts}
\end{solution}
\ifprintanswers\newpage
\else
\fi

\question[2]
Заокружи који од наведених бројева је ирационалан:  
% Иза зареза који није децимални не треба смањити размак.
$$ \variant
{\sqrt{50},\quad 7,\quad \frac23}
{\sqrt{8},\quad -5,\quad 1,\!25}
{\sqrt{3},\quad -2,\quad \frac{5}{4}}
$$
\begin{solution}[\stretch 1]
Ирационални број је:
\variant{$\sqrt{50}$}{$\sqrt{8}$}{$\sqrt{3}$}
\end{solution}

\question[3]
Катете правоуглог троугла имају дужине $\measure{\variant{6}{8}{5}}{cm}$ и $\measure{\variant{8}{15}{12}}{cm}$. Израчунај дужину хипотенузе.
\begin{solution}[\stretch 3]
$$
c = \sqrt{a^2 + b^2} = \sqrt{\variant{6}{8}{5}^2 + \variant{8}{15}{12}^2} = \sqrt{\variant{36+64}{64+225}{25+144}} = \sqrt{\variant{100}{289}{169}} = \measure{\variant{10}{17}{13}}{cm}
$$
\end{solution}
\ifprintanswers
\else\answerline\newpage
\fi

\section{Средњи задаци (стандардни ниво)}

\question[4]
У табели допуни вредности:
$$
\begin{array}{|c|c|c|} \hline
x & x^2 & \sqrt{x^2} \\ \hline
\variant{3}{-4}{\frac{5}{2}} & & \\ \hline
\variant{-7}{\frac{3}{4}}{-1,\!2} & & \\ \hline
\end{array}
$$
\begin{solution}[\stretch 1]
Први ред:
\quad $x^2 = \variant{9}{16}{\frac{25}{4}}$,
\quad $\sqrt{x^2} = \variant{3}{4}{\frac{5}{2}}$

Други ред:
\quad $x^2 = \variant{49}{\frac{9}{16}}{1,\!44}$,
\quad $\sqrt{x^2} = \variant{7}{\frac{3}{4}}{1,\!2}$
\end{solution}
\ifprintanswers\newpage
\else
\fi

\question[4]
Хипотенуза правоуглог троугла има дужину $\measure{\variant{13}{25}{10}}{cm}$, а једна катета $\measure{\variant{5}{20}{6}}{cm}$. Израчунај обим и површину троугла.
\begin{solution}[\stretch 4]
\begin{align*}
b &= \sqrt{c^2 - a^2}
= \sqrt{\variant{13}{25}{10}^2 - \variant{5}{20}{6}^2}
= \sqrt{\variant{169-25}{625-400}{100-36}}
= \sqrt{\variant{144}{225}{64}} = \variant{12}{15}{8}
\\
O &= a + b + c = \variant{5+12+13}{20+15+25}{6+8+10}
= \measure{\variant{30}{60}{24}}{cm}
\\
P &= \frac{a\cdot b}{2}
= \frac{\variant{5}{20}{6}\cdot\variant{12}{15}{8}}{2}
= \measure{\variant{30}{150}{24}}{cm^2}
\end{align*}
\end{solution}
\ifprintanswers
\else\newpage
\fi

\question[3]
Израчунај:
$$
\frac{\sqrt{\variant{49}{81}{25}} - \sqrt{\variant{9}{16}{4}}}{\sqrt{\variant{4}{9}{16}}}
% У основној школи се не учи кубни корен, само квадратни.
$$
\begin{solution}[\stretch 3]
Бројилац: $\sqrt{\variant{49}{81}{25}} - \sqrt{\variant{9}{16}{4}}
= \variant{7-3}{9-4}{5-2} = \variant{4}{5}{3}$

Именилац: $\sqrt{\variant{4}{9}{16}} = \variant{2}{3}{4}$

Цео израз: $\frac{\variant{4}{5}{3}}{\variant{2}{3}{4}}
= \variant{2}{\frac{5}{3}}{\frac{3}{4}}$
\end{solution}
\ifprintanswers\newpage
\else\answerline
\fi

\section{Тежи задаци (изазов)}
% Ученик који полаже поправни испит није показивао значајније
% интересовање за учење током школске године.
% Тешко је мотивисати ученика да прихвати изазов у интелектуалној сфери.

\question[5]
Правоугли троугао има катете $\measure{\variant{9}{15}{7}}{cm}$ и $\measure{\variant{12}{8}{24}}{cm}$. Израчунај полупречник уписаног круга.
\begin{solution}[\stretch 4]
\begin{align*}
c &= \sqrt{a^2 + b^2} = \sqrt{\variant{81+144}{225+64}{49+576}}
= \sqrt{\variant{225}{289}{625}} = \variant{15}{17}{25}
\\
r &= \frac{a+b-c}{2}
= \frac{\variant{9+12-15}{15+8-17}{7+24-25}}{2}
= \frac{\variant{6}{6}{6}}{2} = \measure 3{cm}
\end{align*}
\end{solution}
\ifprintanswers
\else\newpage
\fi

\question[5]
(Домаћи) Израчунај површину троугла са страницама
$\measure{\variant{13}{10}{9}}{cm}$,
$\measure{\variant{14}{15}{10}}{cm}$ и
$\measure{\variant{15}{7}{15}}{cm}$.
\begin{solution}[\stretch 5]
\begin{align*}
s &= \frac{a+b+c}{2}
= \frac{\variant{13+14+15}{10+15+7}{9+10+15}}{2}
= \frac{\variant{42}{32}{34}}{2} = \variant{21}{16}{17}
\\
P &= \sqrt{s(s-a)(s-b)(s-c)} = \sqrt{\variant
	{21\cdot 8\cdot 7\cdot 6}
	{16\cdot 6\cdot 1\cdot 9}
	{17\cdot 8\cdot 7\cdot 2}}
\\
P &= \sqrt{\variant
	{9\cdot 16\cdot 49}
	{16\cdot 9\cdot 6}
	{17\cdot 7\cdot 8\cdot 2}}
= \measure{\variant{84}{12\sqrt{6}}{4\sqrt{119}}}{cm^2}
\end{align*}
\end{solution}

\end{questions}

\end{document}
