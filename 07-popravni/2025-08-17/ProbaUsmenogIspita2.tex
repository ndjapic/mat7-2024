\documentclass[11pt,a5paper,twoside,addpoints,answers]{exam}

\usepackage[OT2]{fontenc}
\usepackage[utf8x]{inputenc}
\usepackage[serbian]{babel}
\usepackage{amssymb,amsmath,geometry}
\geometry{a5paper, margin=1.5cm}

% Јединице мере
\newcommand{\measure}[2]{#1\,\mathrm{#2}}

% Макро за 3 варијанте
\newcommand{\variant}[3]{#1}

% Подешавање
\renewcommand{\solutiontitle}{\noindent\textbf{Концепт:}\enspace}
\cfoot[]{Страна \thepage\ од \numpages}
\pointsinmargin
\pointname{}

\title{Усмени део поправног испита}
\author{Проба}
\date{Станишић, 17.~август 2025.}

\begin{document}
\maketitle

\begin{questions}

% ============================
% Комбинација 1
% ============================
\fullwidth{\large\textbf{Комбинација 1}}
\question Објасни разлику између тетиве, пречника и тангенте у кругу. Наведи и докажи бар једно тврђење које их повезује.
\begin{solution}
Дефиниције; пречник пролази кроз центар; тангента нормална на полупречник у додирној тачки. Доказ: нормала из центра на тетиву је симетрала.
\\[0.5em]
\textit{Смернице за подпитања:}  
- Шта је заједничко тетиви и пречнику?  
- Како се доказује да је тангента нормална на полупречник?  
- Да ли свака права која пролази кроз центар мора бити пречник?  
\end{solution}

\question Наведи и докажи формулу за број дијагонала у конвексном $n$-углу.
\begin{solution}
$D=\dfrac{n(n-3)}2$. Свака од $n$ темена спаја се са $n-3$ темена; свака дијагонала бројана два пута.  
\\[0.5em]
\textit{Смернице за подпитања:}  
- Колико дијагонала има петоугао? Шестоугао?  
- Да ли формула важи и за $n=3$ или $n=4$?  
\end{solution}

\question Израчунај површину кружног исечка полупречника $r=\measure{6}{cm}$ за угао $120^\circ$.
\begin{solution}
$P=\tfrac{120}{360}\pi r^2=12\pi\ \mathrm{cm}^2$.  
\\[0.5em]
\textit{Смернице за подпитања:}  
- Како се израчунава површина целог круга?  
- Који је однос између угла у центру и површине исечка?  
\end{solution}

% ============================
% Комбинација 2
% ============================
\fullwidth{\large\textbf{Комбинација 2}}
\question Дефиниши правилан многоугао. Објасни поступак израчунавања мере сваког унутрашњег угла правилног $n$-угла.
\begin{solution}
Дефиниција: једнакостраничан и једнакокраки. Сума углова $(n-2)\cdot180^\circ$, угао $=S/n$.  
\\[0.5em]
\textit{Смернице за подпитања:}  
- Шта је правилан троугао? Правилан четвороугао?  
- Како расте мера унутрашњег угла са $n$?  
\end{solution}

\question Наведи и објасни бар један случај подударности троуглова. У којим задацима се најчешће користи?
\begin{solution}
На пример СУС. Објашњење услова и примене.  
\\[0.5em]
\textit{Смернице за подпитања:}  
- Који су још случајеви подударности?  
- У ком задатку из геометрије си користио СУС?  
\end{solution}

\question Израчунај дужину лука кружнице полупречника $\measure{10}{cm}$ за угао $45^\circ$.
\begin{solution}
$l=\frac{45}{360}\cdot 2\pi r=2.5\pi$ cm.  
\\[0.5em]
\textit{Смернице за подпитања:}  
- Која је формула за обим круга?  
- Како би се израчунао лук за угао $180^\circ$?  
\end{solution}

% ============================
% Комбинација 3
% ============================
\fullwidth{\large\textbf{Комбинација 3}}
\question Објасни поступак растављања полинома на чиниоце. Наведи бар два метода.
\begin{solution}
Заједнички множилац; формуле $(a+b)^2$, $a^2-b^2$.  
\\[0.5em]
\textit{Смернице за подпитања:}  
- Како препознати заједнички множилац?  
- Када се користи разлика квадрата?  
\end{solution}

\question Наведи и докажи теорему о збиру унутрашњих углова троугла.
\begin{solution}
Сума $180^\circ$. Доказ преко паралеле.  
\\[0.5em]
\textit{Смернице за подпитања:}  
- Како се може применити у четвороуглу?  
- Који угао се назива спољашњи угао троугла?  
\end{solution}

\question Конструиши тангенту на кружницу $k(O,r)$ у тачки $T$.
\begin{solution}
Спојити $O$ и $T$; конструисати нормалу на $OT$ у $T$; добијена права је тангента.  
\\[0.5em]
\textit{Смернице за подпитања:}  
- Зашто је тангента нормална на полупречник?  
- Како би конструисао прав угао лењиром и шестаром?  
\end{solution}

% ============================
% Комбинација 4
% ============================
\fullwidth{\large\textbf{Комбинација 4}}
\question Објасни шта је кружни исечак и кружни прстен. Наведи примене.
\begin{solution}
Исечак – део круга између два полупречника; прстен – између две концентричне кружнице.  
\\[0.5em]
\textit{Смернице за подпитања:}  
- Како се рачуна површина прстена?  
- Да ли је пречник утицајан код исечка?  
\end{solution}

\question Дефиниши и објасни симетралe троугла. Где се секу?
\begin{solution}
Симетрала дели угао на два једнака; три симетрале се секу у центру уписане кружнице.  
\\[0.5em]
\textit{Смернице за подпитања:}  
- Где се секу висине троугла?  
- Шта је тежиште?  
\end{solution}

\question У квадрат страница $a=6$ уписана је кружница, а око њега описана друга. Израчунај површину прстена.
\begin{solution}
$r_u=3$, $r_o=3\sqrt2$. Површина $=\pi(18-9)=9\pi$.  
\\[0.5em]
\textit{Смернице за подпитања:}  
- Како се добија $r_o$ из $a$?  
- Да ли се овај прстен може поделити на четири једнака дела?  
\end{solution}

% ============================
% Комбинација 5
% ============================
\fullwidth{\large\textbf{Комбинација 5}}
\question Шта је нормала на тетиву у кругу? Докажи да је и симетрала.
\begin{solution}
Нормала из центра на тетиву дели тетиву на два једнака дела. Доказ преко подударности.  
\\[0.5em]
\textit{Смернице за подпитања:}  
- Шта је тетива?  
- Како доказујемо подударност?  
\end{solution}

\question Објасни теорему о висинама у троуглу. Где се секу?
\begin{solution}
Све висине се секу у ортоцентру.  
\\[0.5em]
\textit{Смернице за подпитања:}  
- Како конструисати висину у троуглу?  
- Да ли ортоцентар увек лежи унутар троугла?  
\end{solution}

\question Израчунај дужину тетиве у кругу полупречника $r=13$ ако је растојање од центра до тетиве $d=5$.
\begin{solution}
Полутетива $=\sqrt{13^2-5^2}=12$, па $24$.  
\\[0.5em]
\textit{Смернице за подпитања:}  
- Који троугао користимо у решавању?  
- Зашто смо користили Питагорину теорему?  
\end{solution}

\end{questions}
\end{document}
