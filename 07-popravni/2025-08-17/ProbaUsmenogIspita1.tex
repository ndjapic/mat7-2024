\documentclass[10pt,a5paper,twoside,addpoints,answers]{exam}

\usepackage[OT2]{fontenc}
\usepackage[utf8x]{inputenc}
\usepackage[serbian]{babel}
\usepackage{amssymb,amsmath,geometry}
\geometry{a5paper, margin=1.5cm}

% Јединице мере
\newcommand{\measure}[2]{#1\,\mathrm{#2}}

% Макро за 3 варијанте
\newcommand{\variant}[3]{#1}

% Подешавање
\renewcommand{\solutiontitle}{\noindent\textbf{Концепт:}\enspace}
\cfoot[]{Страна \thepage\ од \numpages}
\pointsinmargin
\pointname{}

\title{Усмени део поправног испита}
\author{Проба}
\date{Станишић, 17.~август 2025.}

\begin{document}
\maketitle

\begin{questions}

% ============================
% Комбинација 1
% ============================
\fullwidth{\large\textbf{Комбинација 1}}
\question Објасни разлику између тетиве, пречника и тангенте у кругу. Наведи и докажи бар једно тврђење које их повезује.
\begin{solution}
Учeник треба да истакне: дефиниције, однос центра и тетиве, да пречник пролази кроз центар, тангента нормална на полупречник у додирној тачки. Доказ: нормала из центра на тетиву је симетрала.
\end{solution}

\question[--] Наведи и докажи формулу за број дијагонала у конвексном $n$-углу. Објасни поступак доказивања.
\begin{solution}
Учeник: $D=\dfrac{n(n-3)}2$. Објашњење: свака од $n$ темена спаја се са $n-3$ темена, добија се $n(n-3)$ дужи, сваку дијагоналу смо пребројали два пута.
\end{solution}

\question[--] Израчунај површину кружног исечка полупречника $r=\measure{6}{cm}$ за угао $120^\circ$.
\begin{solution}
$P=\frac{120}{360}\pi r^2=\tfrac13\pi\cdot36=12\pi\ \mathrm{cm}^2$.
\end{solution}

% ============================
% Комбинација 2
% ============================
\fullwidth{\large\textbf{Комбинација 2}}
\question[--] Дефиниши правилан многоугао. Објасни поступак израчунавања мере сваког унутрашњег угла правилног $n$-угла.
\begin{solution}
Дефиниција: једнакостраничан и једнакокраки. Поступак: $S=(n-2)\cdot180^\circ$, $\alpha=S/n$.
\end{solution}

\question[--] Наведи и објасни бар један случај подударности троуглова. У којим задацима се најчешће користи?
\begin{solution}
На пример, СУС. Учeник објашњава услове и пример примене.
\end{solution}

\question[--] Израчунај дужину лука кружнице полупречника $\measure{10}{cm}$ за угао $45^\circ$.
\begin{solution}
$l=\frac{45}{360}\cdot 2\pi\cdot 10=\frac18\cdot20\pi=\tfrac{20\pi}8=2.5\pi$ cm.
\end{solution}

% ============================
% Комбинација 3
% ============================
\fullwidth{\large\textbf{Комбинација 3}}
\question[--] Објасни поступак растављања полинома на чиниоце. Наведи бар два примера различитих метода.
\begin{solution}
Заједнички множилац; формуле разлике квадрата, квадрата збира/разлике. Учeник треба да покаже поступак.
\end{solution}

\question[--] Наведи и докажи теорему о збиру унутрашњих углова троугла.
\begin{solution}
$180^\circ$. Доказ преко паралеле кроз темe троугла или употребом паралелних углова.
\end{solution}

\question[--] Конструиши тангенту на кружницу $k(O,r)$ у тачки $T$ на кружници.
\begin{solution}
Кораци: спојити $O$ и $T$; конструисати нормалу у $T$ на $OT$; добија се тражена тангента.
\end{solution}

% ============================
% Комбинација 4
% ============================
\fullwidth{\large\textbf{Комбинација 4}}
\question[--] Објасни шта је кружни исечак и кружни прстен. У којим задацима се јављају?
\begin{solution}
Учeник: кружни исечак – део круга одређен углом у центру; кружни прстен – подручје између две концентричне кружнице. Примери.
\end{solution}

\question[--] Дефиниши и објасни симетралe троугла. Где се оне секу?
\begin{solution}
Симетрала – права која дели угао на половине. Све три се секу у центру уписане кружнице.
\end{solution}

\question[--] У квадрат страница $a=6$ уписана је кружница, а затим око квадрата описана друга кружница. Израчунај површину кружног прстена.
\begin{solution}
$r_u=3$, $r_o=3\sqrt2$. Површина $=\pi(r_o^2-r_u^2)=\pi(18-9)=9\pi$.
\end{solution}

% ============================
% Комбинација 5
% ============================
\fullwidth{\large\textbf{Комбинација 5}}
\question[--] Шта је нормала на тетиву у кругу? Докажи да је то уједно и симетрала тетиве.
\begin{solution}
Учeник објашњава и доказује преко подударности два троугла.
\end{solution}

\question[--] Објасни теорему о висинама у троуглу. Где се оне секу?
\begin{solution}
Све висине се секу у ортоцентру. Учeник наводи и доказује или образлаже.
\end{solution}

\question[--] Израчунај дужину тетиве у кругу полупречника $r=13$ ако је растојање од центра до тетиве $d=5$.
\begin{solution}
Полутетива $=\sqrt{r^2-d^2}=\sqrt{169-25}=12$. Тетива $=24$.
\end{solution}

\end{questions}
\end{document}
