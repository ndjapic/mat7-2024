\documentclass[11pt,a5paper,twoside,addpoints,noanswers]{exam} % задаци
%\documentclass[10pt,a5paper,twoside,addpoints,answers]{exam} % решења

\usepackage[OT2]{fontenc}
\usepackage[utf8x]{inputenc}
\usepackage[serbian]{babel}
\usepackage{multicol}
\usepackage{amssymb,amsmath}
\usepackage{graphicx}
\usepackage{geometry}
\geometry{a5paper, margin=1.5cm}

% Макро за јединице мере
\newcommand{\measure}[2]{#1\,\mathrm{#2}}

% Макро за варијанте
\newcommand{\variant}[4]{#1}

% Подешавања за exam
%\printanswers % <- откоментаришите да бисте видели решења
\renewcommand{\solutiontitle}{\noindent\textrm{Решење:}\enspace}
\pointsinmargin
\pointname{}
\hqword{Задатак:}
\hpgword{Страница:}
\hpword{Поени:}
\hsword{Остварено:}
\htword{Збир}
\vqword{Задатак:}
\vpgword{Страница:}
\vpword{Поени:}
\vsword{Остварено:}
\vtword{Збир}
\cellwidth{1em}
\cfoot[]{Страница \thepage\ од \numpages}
\addto{\captionsserbian}{\renewcommand{\abstractname}{Упутство}}

\title{Припремна настава}
\author{$\mathrm{VII}_\Box$ Примена Питагорине теореме. Степеновање
 \thanks{
  33 одлично,
  25 врло добро,
  17 добро,
   9 довољно.
 }
}
\date{Станишић, 12.\ август 2025.}

\pagestyle{headandfoot}
%\firstpageheader{Примена Питагорине теореме. Степеновање}
%	{}{варијанта \variant 1234}
\runningheader{Примена Питагорине теореме. Степеновање}
	{}{варијанта \variant 1234}
%\firstpagefooter{Припремна настава}
%	{}{Страна \thepage\ од \numpages}
\runningfooter{уторак}
	{}{Страна \thepage\ од \numpages}

\begin{document}
\maketitle
\thispagestyle{headandfoot}

\ifprintanswers\else
\begin{flushleft}
\gradetable[v]\newpage
\end{flushleft}
\fi

\begin{questions}

\question %01.
Израчунај:
 \begin{multicols}{3}
 \begin{parts}
 \part[1] $2^{\variant{8}{7}{9}{6}}$;
 \part[1] $10^{\variant{6}{7}{5}{6}}$;
 \part[1] $\variant{3^5}{5^4}{3^6}{5^3}$.
 \end{parts}
 \end{multicols}

\begin{solution}[\stretch 1]
%\begin{multicols}{3}
\begin{parts}
\part $2^{\variant{8}{7}{9}{6}}=\variant{256}{128}{512}{64}$.
\part $10^{\variant{6}{7}{5}{6}}=\variant{1\,000\,000}{10\,000\,000}{100\,000}{1\,000\,000}$.
\part $\variant{3^5}{5^4}{3^6}{5^3}=\variant{243}{625}{729}{125}$.
\end{parts}
%\end{multicols}
\end{solution}

\question %02.
Запиши у облику степена:
% \begin{multicols}{2}
 \begin{parts}
 \part[1] \variant{
  $0,\!3 \cdot 0,\!3 \cdot 0,\!3 \cdot 0,\!3 \cdot 0,\!3$
 }{
  $2,\!1 \cdot 2,\!1 \cdot 2,\!1 \cdot 2,\!1$
 }{
  $0,\!6 \cdot 0,\!6 \cdot 0,\!6 \cdot 0,\!6 \cdot 0,\!6$
 }{
  $1,\!4 \cdot 1,\!4 \cdot 1,\!4 \cdot 1,\!4$
 };
 \part[1] $\displaystyle \variant{
  \left( -\frac{4}{3} \right) \cdot
  \left( -\frac{4}{3} \right) \cdot
  \left( -\frac{4}{3} \right) \cdot
  \left( -\frac{4}{3} \right)
 }{
  \frac{\sqrt{6}}{4} \cdot
  \frac{\sqrt{6}}{4} \cdot
  \frac{\sqrt{6}}{4} \cdot
  \frac{\sqrt{6}}{4} \cdot
  \frac{\sqrt{6}}{4} \cdot
  \frac{\sqrt{6}}{4} \cdot
  \frac{\sqrt{6}}{4}
 }{
  \left( -\frac{5}{2} \right) \cdot
  \left( -\frac{5}{2} \right) \cdot
  \left( -\frac{5}{2} \right) \cdot
  \left( -\frac{5}{2} \right) \cdot
  \left( -\frac{5}{2} \right)
 }{
  \sqrt{\frac{3}{7}} \cdot
  \sqrt{\frac{3}{7}} \cdot
  \sqrt{\frac{3}{7}} \cdot
  \sqrt{\frac{3}{7}} \cdot
  \sqrt{\frac{3}{7}}
 }$.
 \end{parts}
% \end{multicols}

\begin{solution}%[\stretch 1]
\begin{multicols}{2}
\begin{parts}
\part $\variant{0,\!3^5}{2,\!1^4}{0,\!6^5}{1,\!4^4}$
\part $\displaystyle \variant{
	\left(-\frac{4}{3}\right)^4
}{
	\left(\frac{\sqrt{6}}{4}\right)^7
}{
	\left(-\frac{5}{2}\right)^5
}{
	\left(\sqrt{\frac{3}{7}}\right)^5
	% = \left(\frac{3}{7}\right)^{5/2}
}$
\end{parts}
\end{multicols}
\end{solution}

\question[3] %03.
Поређај бројеве од најмањег до највећег:
\[
\variant
 {2^7,\quad 5^3,\quad 11^2,\quad 4^3}
 {9^2,\quad 3^4,\, 2^6,\quad 5^3}
 {2^6,\quad 11^2,\quad 5^3,\quad 4^3}
 {3^4,\quad 2^7,\quad 10^2,\quad 5^3}.
\]

\begin{solution}[\stretch 2]
\variant{
	$4^3=64,\; 5^3=125,\; 11^2=121,\; 2^7=128$.
	Редослед: $64 < 121 < 125 < 128$.
	\quad \fbox{$4^3 < 11^2 < 5^3 < 2^7$}.
}{
	$2^6=64,\; 3^4=81,\; 5^3=125,\; 9^2=81$
	-- уочите $9^2$ и $3^4$; оба су 81.
	Редослед: $64 < 81 = 81 < 125$.
	\quad \fbox{$2^6 \leqslant 3^4 \leqslant 9^2 \leqslant 5^3$}.
}{
	$2^6=64,\; 4^3=64,\; 5^3=125,\; 11^2=121$.
	-- уочите $2^6$ и $4^3$; оба су 64.
	Редослед: $64 = 64 < 121 < 125$.
	\quad \fbox{$2^6 \leqslant 4^3 \leqslant 11^2 \leqslant 5^3$}.
}{
	$3^4=81,\; 5^3=125,\; 10^2=100,\; 2^7=128$.
	Редослед: $81 < 100 < 125 < 128$.
	\quad \fbox{$3^4 < 10^2 < 5^3 < 2^7$}.
}
\end{solution}

\question %04.
 Поједностави израз:
 \begin{multicols}{2}
 \begin{parts}
 \part[1] $\displaystyle \variant
  {\frac{x^6}{x^7}}
  {\frac{a^7}{a^5}}
  {\frac{y^9}{y^{10}}}
  {\frac{c^{13}}{c^9}}
 $;
 \part[2] $\displaystyle \variant
  {\frac{t^{14} \cdot t^4}{t^{18}}}
  {\frac{s^{13}}{s^6 \cdot s^9}}
  {\frac{a^{16}}{a^6 \cdot a^{11}}}
  {\frac{x^{11} \cdot x^6}{x^{19}}}
 $.
 \end{parts}
 \end{multicols}

\begin{solution}[\stretch 1]
\begin{multicols}{2}
\begin{parts}
\part $\displaystyle \variant
	{\frac{x^6}{x^7} = x^{-1} = \frac{1}{x}}
	{\frac{a^7}{a^5} = a^2}
	{\frac{y^9}{y^{10}} = y^{-1} = \frac{1}{y}}
	{\frac{c^{13}}{c^9} = c^4}
$.
\part $\displaystyle \variant
	{\frac{t^{14}\cdot t^4}{t^{18}} = t^0 = 1}
	{\frac{s^{13}}{s^6\cdot s^9} = s^{-2} = \frac{1}{s^2}}
	{\frac{a^{16}}{a^6\cdot a^{11}} = a^{-1} = \frac{1}{a}}
	{\frac{x^{11}\cdot x^6}{x^{19}} = x^{-2} = \frac{1}{x^2}}
$.
\end{parts}
\end{multicols}
\end{solution}
\ifprintanswers\else\newpage\fi

\question %05.
 Израчунај вредност израза
 \begin{multicols}{2}
 \begin{parts}
 \part[1] $\displaystyle \variant
  {\frac{2^{12}}{2^9}}
  {\frac{4^6}{2^7}}
  {\frac{5^7}{5^5}}
  {\frac{3^9}{3^{10}}}
 $;
 \part[3] $\displaystyle \variant
  {\frac{4^8}{2^6 \cdot 8^3}}
  {\frac{4^5 \cdot 8^2}{2^{17}}}
  {\frac{8^4}{4^3 \cdot 2^8}}
  {\frac{8^5 \cdot 2^3}{4^9}}
 $.
 \end{parts}
 \end{multicols}

\begin{solution}[\stretch 2]
\begin{multicols}{2}
\begin{parts}
\part $\displaystyle \variant
	{ \frac{2^{12}}{2^9} = 2^3 = 8 }
	{ \frac{4^6}{2^7}
		= \frac{(2^2)^6}{2^7}
		= \frac{2^{12}}{2^7} = 2^5 = 32 }
	{ \frac{5^7}{5^5} = 5^2 = 25 }
	{ \frac{3^9}{3^{10}} = 3^{-1} = \frac 13 }
$.
\part $\displaystyle \variant{
	\frac{4^8}{2^6 \cdot 8^3}
	= \frac{2^{16}}{2^6 \cdot 2^9}
	= 2^{16-15} = 2^1 = 2
}{
	\frac{4^5 \cdot 8^2}{2^{17}}
	= \frac{2^{10} \cdot 2^6}{2^{17}}
	= 2^{-1} = \frac 12
}{
	\frac{8^4}{4^3 \cdot 2^8}
	= \frac{2^{12}}{2^6 \cdot 2^8}
	= 2^{-2} = \frac 14
}{
	\frac{8^5\cdot 2^3}{4^9}
	= \frac{2^{15}\cdot 2^3}{2^{18}}
	= 2^0 = 1
}$
\end{parts}
\end{multicols}
\end{solution}

\question[3] %06.
 За коју вредност променљиве $x$ је тачна једнакост
 \(
  \variant
   {2^x = 4^6}
   {(6^{2x})^3 = 6^{48}}
   {(5^2)^{3x} = 5^{30}}
   {8^x = 2^{27}}
 \)?
 
\begin{solution}[\stretch 2]
$\variant
	{2^x=4^6=(2^2)^6=2^{12}\Rightarrow x=12}
	{(6^{2x})^3=6^{6x}=6^{48}\Rightarrow 6x=48\Rightarrow x=8}
	{(5^2)^{3x}=5^{6x}=5^{30}\Rightarrow 6x=30\Rightarrow x=5}
	{8^x=(2^3)^x=2^{3x}=2^{27}\Rightarrow 3x=27\Rightarrow x=9}
$.
\end{solution}

\question[3] %07.
 Напиши број \variant{256}{15625}{729}{512}
 као степен са основом \variant{2}{5}{3}{2},
 а затим као степен са основом \variant{4}{25}{9}{8}.

\begin{solution}[\stretch 2]
Растављање на просте чиниоце:
$\variant{256 = 2^8}{15625 = 5^6}{729 = 3^6}{512 = 2^9}$.
Са основом $\variant{4}{25}{9}{8}$:
$\variant{256 = 4^4}{15625 = 25^3}{729 = 9^3}{512 = 8^3}$, јер је
$$
	\variant{
		(2 \cdot 2) \cdot (2 \cdot 2) \cdot
		(2 \cdot 2) \cdot (2 \cdot 2) = 256
	}{
		(5 \cdot 5) \cdot
		(5 \cdot 5) \cdot
		(5 \cdot 5) = 15625
	}{
		(3 \cdot 3) \cdot
		(3 \cdot 3) \cdot
		(3 \cdot 3) = 729
	}{
		(2 \cdot 2 \cdot 2) \cdot
		(2 \cdot 2 \cdot 2) \cdot
		(2 \cdot 2 \cdot 2) = 512
	}.
$$
\end{solution}

% ------------------------------
% ДОДАТИ ПИТАГОРИНУ: примена на различите фигуре
% ------------------------------

\question[4] %08.
%(Питагорина – једнакокраки троугао)  
Једнакокраки троугао има основицу
$\measure{\variant{10}{12}{14}{8}}{cm}$,
а висина која је нормална на основицу је
$\measure{\variant{8}{9}{10}{6}}{cm}$.
Израчунај дужину крака, обим и површину троугла.

\begin{solution}[\stretch 3]
Означимо крак са $b$. Половина основице је $
	\frac{\measure{\variant{10}{12}{14}{8}}{cm}}2
	=\measure{\variant{5}{6}{7}{4}}{cm}
$. На основу Питагорине теореме:
\[
	b = \sqrt{
		\variant{5}{6}{7}{4}^2 + \variant{8}{9}{10}{6}^2
	}
	= \sqrt{\variant{25+64}{36+81}{49+100}{16+36}}
	= \measure{
		\variant{\sqrt{89}}{\sqrt{117}}{\sqrt{149}}{\sqrt{52}}
	}{cm}.
\]
Обим:
$
	O = \variant{10}{12}{14}{8} + 2b
	= \measure{(
		\variant{10}{12}{14}{8} +
		\variant{2 \sqrt{89}}{2 \sqrt{117}}{2 \sqrt{149}}{4 \sqrt{13}}
	)}{cm}
$.
Површина:
$$
	P = \frac{1}{2}\cdot
	\measure{\variant{10}{12}{14}{8}}{cm} \cdot
	\measure{\variant{8}{9}{10}{6}}{cm}
	= \measure{\variant{40}{54}{70}{24}}{cm^2}.
$$
\end{solution}
\ifprintanswers\else\newpage\fi

\question[4] %09.
%(Питагорина – једнакостранични троугао)  
У једнакостраничном троуглу страница је $\measure{\variant{6}{9}{12}{8}}{cm}$. Израчунај висину, обим и површину троугла.

\begin{solution}[\stretch 3]
Висина у једнакостраничном троуглу: $h = \frac{\sqrt{3}}{2} a$.
\[
h = \frac{\sqrt{3}}{2}\cdot \measure{\variant{6}{9}{12}{8}}{cm}
= \measure{\variant{3\sqrt{3}}{\tfrac{9\sqrt{3}}{2}}{6\sqrt{3}}{4\sqrt{3}}}{cm}.
\]
Обим: $O = 3\cdot \measure{\variant{6}{9}{12}{8}}{cm} = \measure{\variant{18}{27}{36}{24}}{cm}$.
Површина: $P = \frac{1}{2} a h = \frac{1}{2}\cdot \measure{\variant{6}{9}{12}{8}}{cm} \cdot h
= \measure{\variant{\tfrac{9\sqrt{3}}{2}}{\tfrac{81\sqrt{3}}{4}}{36\sqrt{3}}{16\sqrt{3}}}{cm^2}$.
(Могућ је и бржи начин рачунања: $P=\frac{\sqrt3}{4}a^2$.)
\end{solution}

\question[4] %10.
%(Питагорина – правоугаоник)  
Правоугаоник има странице $\measure{\variant{6}{7}{9}{5}}{cm}$ и $\measure{\variant{8}{24}{12}{12}}{cm}$. Израчунај дужину дијагонале, обим и површину.

\begin{solution}[\stretch 3]
Дијагонала $d=\sqrt{a^2+b^2} = \measure{\sqrt{\variant{6^2+8^2}{7^2+24^2}{9^2+12^2}{5^2+12^2}}}{cm}
= \measure{\variant{10}{25}{15}{13}}{cm}$.
Обим: $O = 2(a+b) = \measure{\variant{28}{62}{42}{34}}{cm}$.
Површина: $P = a\cdot b = \measure{\variant{48}{168}{108}{60}}{cm^2}$.
\end{solution}

\question %11.
%(Питагорина – квадрат/ромб)
\begin{parts}
\part[2]
	Квадрат има дијагоналу $\measure{\variant{10}{12}{8}{14}}{cm}$.
	Израчунај страницу и површину квадрата.
\part[3]
	Ромб има дијагонале
	$d_1 = \measure{\variant{16}{9}{15}{24}}{cm}$ и
	$d_2 = \measure{\variant{12}{12}{20}{10}}{cm}$.
	Израчунај страницу, површину и висину ромба.
\end{parts}

\begin{solution}[\stretch 3]
\begin{parts}
\part
Страница квадрата: $a = \dfrac{d}{\sqrt{2}} = \measure{\dfrac{\variant{10}{12}{8}{14}}{\sqrt{2}}}{cm}$.  
Површина квадрата: $P = a^2 = \measure{\frac{d^2}{2}}{cm^2} = \measure{\variant{50}{72}{32}{98}}{cm^2}$.

\part
За ромб са истим дијагоналама $d_1$ и $d_2$, површина је
$P = \dfrac{d_1 d_2}{2}$ и страница
$$
	 a = \sqrt{
 		\left(
 			\frac{d_1}{2}
 		\right)^2 + \left(
	 		\frac{d_2}{2}
 		\right)^2
 	}.
$$
Висина ромба се израчунава на основу формуле за површину
$P = ah$ израчунавањем непознатог чиниоца $h$.
\end{parts}
\end{solution}

\question[5] %12.
%(Питагорина – трапез / правоугли трапез)  
Правоугли трапез има основице
$\measure{\variant{10}{12}{14}{16}}{cm}$ и
$\measure{\variant{6}{8}{9}{10}}{cm}$
(већа и мања основица) и један од кракова (десни)
$\measure{\variant{6}{5}{8}{6}}{cm}$ нормалан на основицу.
Израчунај висину, други крак, обим и површину трапеза.

\begin{solution}[\stretch 4]
Нека су $a$ и $c$ основице (већа је $a$), $b$ десни крак (нормалан на основице), $d$ леви крак, $h$ висина. Пошто је десни крак нормалан на основице, $h=b=\measure{\variant{6}{5}{8}{6}}{cm}$. Одстојање између пројекције мање и већe основице је
$a-c$%; половина разлике је $\frac{a-c}{2}$
. Према Питагориној теореми:
\[
	d = \sqrt{
		(
			\variant{10}{12}{14}{16} -
			\variant{6}{8}{9}{10}
		)^2 + \variant{6}{5}{8}{6}^2
	} = \sqrt{
		\variant{16}{16}{25}{36} + \variant{36}{25}{64}{36}
	} = \sqrt{\variant{52}{\measure{41}{cm}}{\measure{91}{cm}}{72}}
	\variant{= \measure{2 \sqrt{13}}{cm}}
	{}{}{= \measure{6 \sqrt{2}}{cm}}.
\]

Обим:
$$
	O = a + c + b + d
	= \variant{10}{12}{14}{16}
	+ \variant{6}{8}{9}{10}
	+ \variant{6}{5}{8}{6}
	+ \variant{2 \sqrt{13}}{\sqrt{41}}{\sqrt{91}}{6 \sqrt{2}}
	= \measure{
		\big(
			\variant{22}{25}{31}{32}
			+ \variant{2 \sqrt{13}}{\sqrt{41}}{\sqrt{91}}{6 \sqrt{2}}
		\big)
	}{cm}.
$$

Површина:
$$
	P = \frac{a+c}{2}\cdot h
	= \frac{
		\variant{10}{12}{14}{16} +
		\variant{6}{8}{9}{10}
	}{2} \cdot \variant{6}{5}{8}{6}
	= \variant{8}{10}{23}{13} \cdot \variant{6}{5}{4}{6}
	= \measure{\variant{48}{50}{92}{78}}{cm^2}.
$$
%(У конкретној варијанти унутар $\variant$ убаци вредности и поједностави.)
\end{solution}

\end{questions}

\end{document}
