%================= Пробни поправни испит — 3 варијанте =================
% answers | noanswers
\documentclass[11pt,a5paper,twoside,addpoints,answers]{exam}

\usepackage[OT2]{fontenc}
\usepackage[utf8x]{inputenc}
\usepackage[serbian]{babel}
\usepackage{amssymb,amsmath,graphicx,multicol}
\usepackage{geometry}
\geometry{a5paper, margin=1.5cm}

% Јединице мере
\newcommand{\measure}[2]{#1\,\mathrm{#2}}

% Макро за 3 варијанте
\newcommand{\variant}[3]{#1}

% ---------------- exam подешавања ----------------
\renewcommand{\solutiontitle}{\noindent\textrm{Решење:}\enspace}
\pointsinmargin
\pointname{}
\hqword{Задатак:}
\hpgword{Страница:}
\hpword{Поени:}
\hsword{Остварено:}
\htword{Збир}
\vqword{Задатак:}
\vpgword{Страница:}
\vpword{Поени:}
\vsword{Остварено:}
\vtword{Збир}
\cellwidth{0.7em}
\cfoot[]{Страна \thepage\ од \numpages}

\title{Пробни поправни испит}
\author{$\mathrm{VII}_\Box$ \; Полиноми 1–2, Многоугао, Круг}
\date{Време израде: 45 минута \quad (формат А5 — 4 стране)}

\pagestyle{headandfoot}
\runningheader{Пробни поправни испит}{}{варијанта \variant{1}{2}{3}}
\runningfooter{петак}{}{Страна \thepage\ од \numpages}

\begin{document}
\maketitle
\thispagestyle{headandfoot}

\ifprintanswers\else
\begin{flushleft}\scriptsize
\gradetable[h]
\end{flushleft}
\begin{abstract}\small
Пажљиво читај текст задатка. Наводи поступак (кораке) и своди полиноме на сређен облик.  
Код круга се тражи \emph{тачна} вредност (са $\pi$), осим ако у задатку пише приближна вредност $\pi$.
\end{abstract}
\fi

% ===== Границе за оцене (укупно 28 поена) =====
\noindent\textbf{Бодова:} \numpoints\;=\;28 \quad
\textbf{Оцене:} довољан $\ge 10$;\; добар $\ge 16$;\; врло добар $\ge 22$;\; одличан $\ge 26$.

\begin{questions}

% =========================================================
% 1. Квадрат бинома
% =========================================================
\question[3]
Примени формулу за квадрат бинома и среди израз:
\[
\variant{(3x-2)^2}{(2a+5)^2}{(4y-1)^2}.
\]
\begin{solution}[\stretch 2]
$(u\pm v)^2=u^2\pm2uv+v^2$,
\[
\variant{9x^2-12x+4}{4a^2+20a+25}{16y^2-8y+1}.
\]
\end{solution}

% =========================================================
% 2. Растављање
% =========================================================
\question[3]
Растави на чиниоце:
\[
\variant{6a^2+9a}{9m^2-4}{9x^2-12x+4}.
\]
\begin{solution}[\stretch 2]
\variant{3a(2a+3)}{(3m-2)(3m+2)}{(3x-2)^2}.
\end{solution}

% ---------------------------------------------------------
\ifprintanswers\else\newpage\fi
% ---------------------------------------------------------

% =========================================================
% 3. Полиноми – рачунање и сређивање
% =========================================================
\question[4]
Израчунај и среди:
\[
\variant{
(2x-3)(x+4)-x(x-1)
}{
(3a+2)(a-1)+ (a-4)(a+4)
}{
(5y-1)(y-2)- (2y-3)^2
}.
\]
\begin{solution}[\stretch 3]
\variant{
$2x^2+8x-3x-12-(x^2-x)=x^2+6x-12$
}{
$3a^2-3a+2a-2+(a^2-16)=4a^2-a-18$
}{
$5y^2-10y-y+2-(4y^2-12y+9)=y^2+y-7$.
}
\end{solution}

% =========================================================
% 4. Многоугао
% =========================================================
\question[4]
\variant{
За $n=12$ израчунај број дијагонала и збир унутрашњих углова конвексног $n$-угла.
}{
За $n=15$ израчунај број дијагонала и збир унутрашњих углова конвексног $n$-угла.
}{
Правилан десетоугао: израчунај број дијагонала и меру сваког унутрашњег угла.
}
\begin{solution}[\stretch 3]
$D=\dfrac{n(n-3)}2,\quad S=(n-2)\cdot180^\circ$.
\[
\variant{
D= \frac{12\cdot 9}{2}=54,\quad S=10\cdot180^\circ=1800^\circ
}{
D=\frac{15\cdot12}{2}=90,\quad S=13\cdot180^\circ=2340^\circ
}{
D=\frac{10\cdot7}{2}=35,\;
S=8\cdot180^\circ=1440^\circ,\;
\widehat{\alpha}=\frac{S}{10}=144^\circ.
}
\]
\end{solution}

% ---------------------------------------------------------
\ifprintanswers\else\newpage\fi
% ---------------------------------------------------------

% =========================================================
% 5. Круг — лук/исечак
% =========================================================
\question[4]
\variant{
Круг има полупречник $\measure{7}{cm}$. Израчунај \emph{тачну} површину кружног исечка за угао $120^\circ$.
}{
Круг има полупречник $\measure{8}{cm}$. Израчунај \emph{тачну} дужину лука за угао $135^\circ$.
}{
Круг има полупречник $\measure{6}{cm}$. Израчунај \emph{тачну} дужину лука за угао $30^\circ$.
}
\begin{solution}[\stretch 3]
\variant{
$P_{is}=\frac{120}{360}\pi r^2=\frac{1}{3}\pi\cdot 49=\frac{49}{3}\pi\ \mathrm{cm}^2$
}{
$l=\frac{135}{360}\cdot 2\pi r=\frac{3}{8}\cdot 16\pi=6\pi\ \mathrm{cm}$
}{
$l=\frac{30}{360}\cdot 2\pi r=\frac{1}{12}\cdot 12\pi=\pi\ \mathrm{cm}$.
}
\end{solution}

% =========================================================
% 6. Кратак доказ (подударност)
% =========================================================
\question[3]
\variant{
(Доказ) У једнакокраком троуглу $ABC$ са $AB=AC$ докажи да су $\angle B$ и $\angle C$ једнаки.
}{
(Доказ) У кружници нормала из центра на тетиву пресеца тетиву на половине.
}{
(Доказ) У једнакокраком троуглу $ABC$ ($AB=AC$) медијана $AM$ на основу $BC$ је уједно и нормала и симетрала угла при $A$.
}
\begin{solution}[\stretch 3]
\variant{
Посматрај троуглове $ABC$ и $ACB$.
$AB=AC$ (дато), $AC=AB$ (дато), $BC$ заједничка.
По \textbf{ССС}: $\triangle ABC\cong\triangle ACB$, па су одговарајући углови једнаки: $\angle B=\angle C$.
}{
Нека је $O$ центар, $AB$ тетива и $H$ подножје нормале $OH$ на $AB$.
Троуглови $OHA$ и $OHB$ су прави, имају $OH$ заједничку катету и $OA=OB$ (полупречници). По \textbf{ССС} су подударни, па је $HA=HB$.
}{
Нека је $M$ средиште $BC$. Троуглови $ABM$ и $ACM$ имају $AB=AC$ (дато), $BM=CM$ (дефиниција средишта), $AM$ заједничку. По \textbf{ССС} су подударни, па су $\angle AMB=\angle AMC$ (прави — збир $180^\circ$ и једнаки), те је $AM\perp BC$, а такође $\angle BAM=\angle MAC$ — $AM$ је симетрала угла.
}
\end{solution}

% ---------------------------------------------------------
\ifprintanswers\else\newpage\fi
% ---------------------------------------------------------

% =========================================================
% 7. Конструкција (лењир и шестар)
% =========================================================
\question[3]
\variant{
\textbf{Конструиши} тангенту на кружницу $k(O,r)$ у задатој тачки $T$ на кружници.
}{
\textbf{Конструиши} тетиву $AB$ кружнице $k(O,r)$ тако да је дата унутрашња тачка $M$ средиште тетиве.
}{
\textbf{Конструиши} тангенте из спољашње тачке $P$ на кружницу $k(O,r)$.
}
\begin{solution}[\stretch 5]
\variant{
\emph{Кораци:} (1) Спој $O$ и $T$. (2) На прави кроз $T$ конструиши нормалу на $OT$ (нпр. помоћу симетрала угла). (3) Та права је тражена тангента.  
\emph{Образложење:} Полупречник је нормалан на тангенту у додирној тачки.
}{
\emph{Кораци:} (1) Нацртај праву $OM$. (2) Кроз $M$ конструиши нормалу на $OM$. (3) Пресеци те праве са кружницом су тражени крајеви $A,B$.  
\emph{Образложење:} Нормала из центра на тетиву је њена симетрала; ако $OM\perp AB$ онда је $M$ средина $AB$.
}{
\emph{Кораци:} (1) Спој $O$ и $P$. (2) Нађи средиште $S$ дужи $OP$ (симетрала). (3) Опиši кружницу $k_S(S,SO)$; њени пресечни положаји са $k$ су додирне тачке $T_1,T_2$. (4) Праве $PT_1,PT_2$ су тражене тангенте.  
\emph{Образложење:} $ST_1=SO=SP$, па је $\angle OT_1P$ прав (Талесова теорема), те $PT_1\perp OT_1$.
}
\end{solution}

% =========================================================
% 8. Круг / полиноми – један рачунски
% =========================================================
\question[4]
\variant{
Круг полупречника $\measure{13}{cm}$ има тетиву $AB$ чије је средиште на растојању $\measure{5}{cm}$ од центра.
Израчунај дужину $AB$.
}{
У квадрат страница $\measure{8}{cm}$ уписана је кружница, а затим око квадрата описана кружница.
Израчунај \emph{тачну} површину кружног прстена између њих.
}{
Тачка $P$ је на растојању $\measure{13}{cm}$ од центра круга полупречника $\measure{5}{cm}$.
Израчунај дужину тангентне дужи из $P$ на круг.
}
\begin{solution}[\stretch 3]
\variant{
Полутетива $x=\sqrt{r^2-d^2}=\sqrt{13^2-5^2}=12$, па је $AB=2x=24\ \mathrm{cm}$.
}{
За квадрат $a=8$: $r_u=\tfrac a2=4$, $r_o=\tfrac{a\sqrt2}{2}=4\sqrt2$.  
$P=\pi(r_o^2-r_u^2)=\pi(32-16)=16\pi\ \mathrm{cm}^2$.
}{
У $\triangle OPT$: $PT=\sqrt{OP^2-OT^2}=\sqrt{13^2-5^2}=12\ \mathrm{cm}$.
}
\end{solution}

\end{questions}

\end{document}
