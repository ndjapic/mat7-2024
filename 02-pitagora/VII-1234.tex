\documentclass[12pt,a5paper,addpoints]{exam}

\usepackage{myPaper}
\usepackage{multicol}
\usepackage{amssymb,amsmath}
\usepackage{epsf}

%\usepackage[serbian]{babel}
%\usepackage[OT2]{fontenc}
%\usepackage[utf8]{inputenc}

%\usepackage{ucs}
\usepackage[OT2]{fontenc}
\usepackage[utf8x]{inputenc}
\usepackage[serbian]{babel}
%\usepackage{t2}

\def\grupa#1#2#3#4{#1}
\title{$\mathrm{VII}_\Box$, група \grupa 1234}
\author{Први писмени задатак
 \thanks{
  19 одлично,
  15 врло добро,
  10 добро,
   6 довољно.
 }
}
\date{Станишић, 29.\ октобар 2024.}

\printanswers
%\renewcommand{\solutiontitle}{\noindent\textrm{Решење:}\enspace}
\renewcommand{\solutiontitle}{}
\pointsinmargin
%\pointsinrightmargin
\pointname{}
%\marginpointname{\%}
%\pagename{Страница}
\hqword{Задатак:}
\hpgword{Страница:}
\hpword{Поени:}
\hsword{Остварено:}
\htword{Збир}
\cellwidth{1em}
%\gradetablestretch{1.1}
\cfoot[]{Страница \thepage\ од \numpages}

\def\abs|#1|{\left| #1 \right|}

\hyphenation{
}

\begin{document}

\maketitle
\thispagestyle{headandfoot}

%\vspace*{\stretch 1}
\noindent \gradetable[h]

\begin{questions}

\question[1] %01.
 Катете правоуглог троугла су $\grupa tshv$ и $\grupa rdts$.
 Изрази хипотенузу $\grupa hvrd$, преко датих катета.
 \begin{solution}
  $
   \grupa
    {h = \sqrt{t^2 + r^2}}
    {v = \sqrt{s^2 + d^2}}
    {r = \sqrt{h^2 + t^2}}
    {d = \sqrt{v^2 + s^2}}
  $
 \end{solution}

\question[4] %02.
 Препиши и попуни табелу:
 $$
  \begin{array}{|*5{c|}} \hline
   x & \grupa
    {12 & -1,\!3 & \frac 34 & -1 \frac 56}
    {11 & -1,\!4 & \frac 35 & -2 \frac 23}
    {14 & -1,\!1 & \frac 56 & -1 \frac 34}
    {13 & -1,\!2 & \frac 25 & -3 \frac 12}
   \\\hline
   x^2 & \phantom{121} & \phantom{1,\!44} & \phantom{\frac 4{25}} & \phantom{5 \frac 49} \\\hline
   \sqrt{x^2} & & & & \\\hline
  \end{array}
 $$
 \begin{solution}
 $$
  \begin{array}{|*5{c|}} \hline
   x & \grupa
    {12 & -1,\!3 & \frac 34 & -1 \frac 56 = - \frac{11}6}
    {11 & -1,\!4 & \frac 35 & -2 \frac 23 = - \frac 83}
    {14 & -1,\!1 & \frac 56 & -1 \frac 34 = - \frac 74}
    {13 & -1,\!2 & \frac 25 & -3 \frac 12 = - \frac 72}
   \\\hline
   x^2 & \grupa
    {144 & 1,\!69 & \frac 9{16} & \frac{121}{36} = 3 \frac{13}{36}}
    {121 & 1,\!96 & \frac 9{25} & \frac{64}9 = 7 \frac 19}
    {196 & 1,\!21 & \frac{25}{36} & \frac{49}{16} = 3 \frac 1{16}}
    {169 & 1,\!44 & \frac 4{25} & \frac{49}4 = 12 \frac 14}
   \\\hline
   \sqrt{x^2} & \grupa
    {12 & 1,\!3 & \frac 34 & 1 \frac 56}
    {11 & 1,\!4 & \frac 35 & 2 \frac 23}
    {14 & 1,\!1 & \frac 56 & 1 \frac 34}
    {13 & 1,\!2 & \frac 25 & 3 \frac 12}
   \\\hline
  \end{array}
 $$
 \end{solution}

\question[3] %03.
 \grupa{Хипотенуза}{Једна катета}{Хипотенуза}{Једна катета}
 правоуглог троугла је $\grupa{25}{15}{17}{16} \, \mathrm{cm}$ а
 \grupa{једна катета}{хипотенуза}{једна катета}{хипотенуза} је
 $\grupa{20}{25}{15}{20} \, \mathrm{cm}$.
 Израчунај обим и површину тог троугла.
 \begin{solution}
 \begin{gather*}
  \grupa caca = \grupa{25}{15}{17}{16} \, \mathrm{cm}, \qquad
  \grupa acac = \grupa{20}{25}{15}{20} \, \mathrm{cm}
  \\ b^2 = c^2 - a^2
   = \grupa{25}{25}{17}{20}^2 - \grupa{20}{15}{15}{16}^2
   = \grupa{625}{625}{289}{400} - \grupa{400}{225}{225}{256}
   = \grupa{225}{400}{64}{144} \, \mathrm{cm}^2
  \\ b
   = \sqrt{\grupa{225}{400}{64}{144}}
   = \grupa{15}{20}{8}{12} \, \mathrm{cm}
  \\ {\cal O} = a+b+c
   = \grupa{20+15+25}{15+20+25}{15+8+17}{16+12+20}
   = \grupa{60}{60}{40}{48} \, \mathrm{cm}
  \\ P = \frac{ab}2
   = \frac{\grupa{20⋅15}{15⋅20}{15⋅8}{16⋅12}}2
   = \grupa{10⋅15}{15⋅10}{15⋅4}{8⋅12}
   = \grupa{150}{150}{60}{96} \, \mathrm{cm}²
 \end{gather*}
 \end{solution}

\question[3] %04.
 За $\grupa 8654 \, \mathrm{kg}$ јабука је плаћено
 $\grupa{688}{564}{390}{324}$ динара.
 Колико треба платити за
 $\grupa 3487 \, \mathrm{kg}$ јабука?
 \begin{solution}
  Први начин:
  \begin{gather*}
   \grupa{688:8}{564:6}{390:5}{324:4}
   = \grupa{(640+48):8}{(540+24):6}{(350+40):5}{(320+4):4}
   = \grupa{86}{94}{78}{81} \, \text{дин} / \text{kg}
   \\ \grupa{86⋅3}{94⋅4}{78⋅8}{81⋅7}
   = \grupa{240+18}{360+16}{560+64}{560+7}
   = \fbox{\grupa{258}{376}{624}{567}\,\text{дин}}
   \text{ за } \grupa 3487 \, \text{kg} \text{ јабука}
  \end{gather*}
  Други начин:
  \begin{gather*}
   \grupa 8654 : \grupa{688}{564}{390}{324} = \grupa 3487 : x
   \\ \grupa 8654 ⋅ x = \grupa{688}{564}{390}{324} ⋅ \grupa 3487
   \\ x = \frac{\grupa{688}{564}{390}{324} ⋅ \grupa 3487}{\grupa 8654}
    = \frac{\grupa{86⋅3}{94⋅4}{78⋅8}{81⋅7}}1
    = \fbox{\grupa{258}{376}{624}{567}\,\text{дин}}
  \end{gather*}
 \end{solution}

\newpage

\question %05.
 Између којих узастопних целих бројева су ирационални бројеви:
% \begin{multicols}2
 \begin{parts}
 \part[1] $\sqrt{\grupa{19}{15}{26}{24}}$;
 \begin{solution}
  \begin{gather*}
   \sqrt{\grupa{16}{9}{25}{16}} <
   \sqrt{\grupa{19}{15}{26}{24}} <
   \sqrt{\grupa{25}{16}{36}{25}} \\
   \grupa{4}{3}{5}{4} <
   \sqrt{\grupa{19}{15}{26}{24}} <
   \grupa{5}{4}{6}{16}
  \end{gather*}
 \end{solution}
 \part[2] $\sqrt{\grupa{17}{24}{15}{26}} - \grupa 3253$?
 \begin{solution}
  \begin{gather*}
   \sqrt{\grupa{16}{16}{9}{25}} <
   \sqrt{\grupa{17}{24}{15}{26}} <
   \sqrt{\grupa{25}{25}{16}{36}} \\
   \grupa{4}{4}{3}{5} <
   \sqrt{\grupa{17}{24}{15}{26}} <
   \grupa{5}{5}{4}{6} \\
   \grupa{4-3}{4-2}{3-5}{5-3} <
   \sqrt{\grupa{17}{24}{15}{26}} - \grupa 3253 <
   \grupa{5-3}{5-2}{4-5}{6-3} \\
   \grupa{1}{2}{-2}{2} <
   \sqrt{\grupa{17}{24}{15}{26}} - \grupa 3253 <
   \grupa{2}{3}{-1}{3}
  \end{gather*}
 \end{solution}
 \end{parts}
% \end{multicols}

\newpage

\question[4] %06.
 Израчунај вредност израза
 $$
  \grupa{
   \frac 23 ⋅ \sqrt{81} - 4 ⋅ \sqrt{\frac{25}{36}} + 6 ⋅ \sqrt{7 \frac 19}
  }{
   \frac 45 ⋅ \sqrt{1 \frac 9{16}} - 3 ⋅ \sqrt{\frac 49} + \frac 23 ⋅ \sqrt{2 \frac 14}
  }{
   \frac 34 ⋅ \sqrt{1 \frac 79} - \frac 12 ⋅ \sqrt{64} + 2 ⋅ \sqrt{0,\!04}
  }{
   \frac 34 ⋅ \sqrt{16} + 9 ⋅ \sqrt{\frac 49} - 6 ⋅ \sqrt{2 \frac 14}
  }.
 $$
 \begin{solution}
 \begin{align*}
  \grupa{
   \frac 23 ⋅ \sqrt{81} - 4 ⋅ \sqrt{\frac{25}{36}} + 6 ⋅ \sqrt{7 \frac 19}
  }{
   \frac 45 ⋅ \sqrt{1 \frac 9{16}} - 3 ⋅ \sqrt{\frac 49} + \frac 23 ⋅ \sqrt{2 \frac 14}
  }{
   \frac 34 ⋅ \sqrt{1 \frac 79} - \frac 12 ⋅ \sqrt{64} + 2 ⋅ \sqrt{0,\!04}
  }{
   \frac 34 ⋅ \sqrt{16} + 9 ⋅ \sqrt{\frac 49} - 6 ⋅ \sqrt{2 \frac 14}
  }
  &= \grupa{
   \frac 23 ⋅ 9 - 4 ⋅ \frac 56 + 6 ⋅ \sqrt{\frac{64}9}
  }{
   \frac 45 ⋅ \sqrt{\frac{25}{16}} - 3 ⋅ \frac 23 + \frac 23 ⋅ \sqrt{\frac 94}
  }{
   \frac 34 ⋅ \sqrt{\frac{16}9} - \frac 12 ⋅ 8 + 2 ⋅ 0,\!2
  }{
   \frac 34 ⋅ 4 + 9 ⋅ \frac 23 - 6 ⋅ \sqrt{\frac 94}
  }
  \\ &= \grupa{
   \frac 23 ⋅ \frac 91 - \frac 41 ⋅ \frac 56 + \frac 61 ⋅ \frac 83
  }{
   \frac 45 ⋅ \frac 54 - \frac 31 ⋅ \frac 23 + \frac 23 ⋅ \frac 32
  }{
   \frac 34 ⋅ \frac 43 - \frac 12 ⋅ \frac 81 + 0,\!4
  }{
   \frac 34 ⋅ \frac 41 + \frac 91 ⋅ \frac 23 - \frac 61 ⋅ \frac 32
  }
  \\ &= \grupa{
   \frac 61 - \frac{10}3 + \frac{16}1
  }{
   \frac 11 - \frac 21 + \frac 11
  }{
   \frac 11 - \frac 41 + 0,\!4
  }{
   \frac 31 + \frac 61 - \frac 91
  }
  \\ &= \grupa{22 - 3\frac 13}{-1+1}{-3 + 0,\!4}{9-9}
  = \grupa{18 \frac 23}{0}{-2,\!6}{0}
 \end{align*}
 \end{solution}

\question[4] %07.
 Израчунај обим и површину ромба чије су дијагонале
 $\grupa{8}{24}{16}{9} \, \mathrm{cm}$ и
 $\grupa{15}{7}{12}{12} \, \mathrm{cm}$.
 \begin{solution}
 $$
 \begin{aligned}
  d_1 = \grupa{8}{24}{16}{9} &\quad \, \mathrm{cm},
  d_2 = \grupa{15}{7}{12}{12} \, \mathrm{cm}
  \\ \grupa
   {(2a)² &= {d_1}² + {d_2}²}
   {(2a)² &= {d_1}² + {d_2}²}
   {a² &= (\frac{d_1}2)² + (\frac{d_2}2)²}
   {(2a)² &= {d_1}² + {d_2}²}
  \\ \grupa
   {(2a)² &= 8² + 15²}
   {(2a)² &= 24² + 7²}
   {a² &= (\frac{16}2)² + (\frac{12}2)²}
   {(2a)² &= 9² + 12²}
  \\ \grupa
   {(2a)² &= 64 + 225}
   {(2a)² &= 576 + 49}
   {a² &= 8² + 6²}
   {(2a)² &= 81 + 144}
  \\ \grupa{(2a)² &= 289}{(2a)² &= 625}{a² &= 64 + 36}{(2a)² &= 225}
  \\ \grupa
   {2a &= \sqrt{289}}
   {2a &= \sqrt{625}}
   {a² &= 100}
   {2a &= \sqrt{225}}
 \end{aligned} \qquad
 \begin{aligned}
  \grupa{2a &= 17}{2a &= 25}{a² &= \sqrt{100}}{2a &= 15}
  \\ a &= \grupa{8,\!5}{12,\!5}{10}{7,\!5} \, \mathrm{cm}
  \\ {\cal O} &= 4a
  = 4 ⋅ \grupa{8,\!5}{12,\!5}{10}{7,\!5}
  = \fbox{\grupa{34}{50}{40}{30}} \, \mathrm{cm}
  \\ P &= \frac{d_1 ⋅ d_2}2
  = \frac{ \grupa{8 ⋅ 15}{24 ⋅ 7}{16 ⋅ 12}{9 ⋅ 12} }2
  \\ P &= \grupa{4 ⋅ 15}{12 ⋅ 7}{8 ⋅ 12}{9 ⋅ 6}
  = \fbox{\grupa{60}{84}{96}{54}} \, \mathrm{cm}²
 \end{aligned}
 $$
 \end{solution}

\end{questions}

\end{document}
%\{} ^2
% $2,\!5 \,\mathrm{cm}$
% \begin{multicols}2
