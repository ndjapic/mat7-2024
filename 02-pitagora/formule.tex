%\documentclass{beamer}
\documentclass[12pt,handout]{beamer}
\usetheme{CambridgeUS}
\usefonttheme{professionalfonts}
\usepackage[OT2]{fontenc}
\usepackage[utf8x]{inputenc}
\usepackage[english,serbian]{babel}
\usepackage{amsmath,amssymb}
\usepackage{tikz}
\usetikzlibrary{positioning}
\usetikzlibrary{graphs,graphs.standard}
\usepackage{forest}
%\usetikzlibrary{graphs,graphdrawing}
%\usegdlibrary{layered}
\usetikzlibrary{graphs, graphs.standard, quotes}% quotes library is for the [""] edges

\title{Примена Питагорине теореме}
\author{систематизација}
\date{\today}

\begin{document}

\begin{frame}
\titlepage
\end{frame}

\begin{frame}{Правоугли троугао}
\begin{tikzpicture}
\onslide<1->
  % Чворови за формуле
  \node[draw, rectangle] (FO) at (0,6) {$O = a+b+c$};
  \node[draw, rectangle] (FP) at (0,4) {$P = \frac{ab}{2}$};
  \node[draw, rectangle] (Fc) at (0,3) {$c^2 = a^2 + b^2$};
  \node[draw, rectangle] (Fro) at (0,2) {$r_o = \frac c2$};
  \node[draw, rectangle] (Fru) at (0,0) {$a+b = c+2r_u$};

\onslide<2->
  % Чворови за променљиве
  \node[draw, circle] (VO) at (3,6) {$O$};
  \node[draw, circle] (VP) at (3,5) {$P$};
  \node[draw, circle] (Va) at (3,4) {$a$};
  \node[draw, circle] (Vb) at (3,3) {$b$};
  \node[draw, circle] (Vc) at (3,2) {$c$};
  \node[draw, circle] (Vro) at (3,1) {$r_o$};
  \node[draw, circle] (Vru) at (3,0) {$r_u$};

\onslide<3->
  % Гране између чворова
  \draw (FO) -- (VO);
  \draw (FO) -- (Va);
  \draw (FO) -- (Vb);
  \draw (FO) -- (Vc);
  \draw (FP) -- (VP);
  \draw (FP) -- (Va);
  \draw (FP) -- (Vb);
  \draw (Fc) -- (Vc);
  \draw (Fc) -- (Va);
  \draw (Fc) -- (Vb);
  \draw (Fro) -- (Vro);
  \draw (Fro) -- (Vc);
  \draw (Fru) -- (Vru);
  \draw (Fru) -- (Va);
  \draw (Fru) -- (Vb);
  \draw (Fru) -- (Vc);
\end{tikzpicture}
\end{frame}

\begin{frame}{Правоугаоник и квадрат}
\begin{tikzpicture}[node distance=1cm, auto]
  % Stilovi za čvorove i grane
  \tikzstyle{formula} = [draw, rectangle]
  \tikzstyle{variable} = [draw, circle]

\onslide<1->
  % Чворови за формуле
  \node[draw, rectangle] (FO) at (0,5) {$O = 2a + 2b$};
  \node[draw, rectangle] (FP) at (0,3) {$P = ab$};
  \node[draw, rectangle] (Fd) at (0,1) {$d^2 = a^2 + b^2$};
  \node[draw, rectangle] (Fr) at (0,0) {$r_o = \frac d2$};

  % Чворови за променљиве
  \node[draw, circle] (VO) at (3,5) {$O$};
  \node[draw, circle] (VP) at (3,4) {$P$};
  \node[draw, circle] (Va) at (3,3) {$a$};
  \node[draw, circle] (Vb) at (3,2) {$b$};
  \node[draw, circle] (Vd) at (3,1) {$d$};
  \node[draw, circle] (Vr) at (3,0) {$r_o$};

\onslide<2->
  % Гране између чворова
  \draw (FO) -- (VO);
  \draw (FO) -- (Va);
  \draw (FO) -- (Vb);
  \draw (FP) -- (VP);
  \draw (FP) -- (Va);
  \draw (FP) -- (Vb);
  \draw (Fd) -- (Vd);
  \draw (Fd) -- (Va);
  \draw (Fd) -- (Vb);
  \draw (Fr) -- (Vd);
  \draw (Fr) -- (Vr);
\end{tikzpicture}
\hfill
\begin{tikzpicture}
\onslide<1->
  % Чворови за формуле
  \node[draw, rectangle] (FO) at (0,5) {$O = 4a$};
  \node[draw, rectangle] (FP) at (0,4) {$P = a^2$};
  \node[draw, rectangle] (Fd) at (0,3) {$d = a \sqrt{2}$};
  \node[draw, rectangle] (Fru) at (0,1) {$r_u = \frac a2$};
  \node[draw, rectangle] (Fro) at (0,0) {$r_o = \frac d2$};

\onslide<2->
  % Чворови за променљиве
  \node[draw, circle] (VO) at (3,5) {$O$};
  \node[draw, circle] (VP) at (3,4) {$P$};
  \node[draw, circle] (Va) at (3,3) {$a$};
  \node[draw, circle] (Vd) at (3,2) {$d$};
  \node[draw, circle] (Vru) at (3,1) {$r_u$};
  \node[draw, circle] (Vro) at (3,0) {$r_o$};

  % Гране између чворова
  \draw (FO) -- (VO);
  \draw (FO) -- (Va);
  \draw (FP) -- (VP);
  \draw (FP) -- (Va);
  \draw (Fd) -- (Vd);
  \draw (Fd) -- (Va);
  \draw (Fro) -- (Vro);
  \draw (Fro) -- (Vd);
  \draw (Fru) -- (Vru);
  \draw (Fru) -- (Va);
\end{tikzpicture}
\end{frame}

\begin{frame}{Једнакокраки и једнакостранични троугао}
\begin{tikzpicture}
\onslide<1->
  % Чворови за формуле
  \node[draw, rectangle] (FO) at (0,4) {$O = a+2b$};
  \node[draw, rectangle] (FP) at (0,2) {$P = \frac{ah}{2}$};
  \node[draw, rectangle] (Fh) at (0,0) {$h² = b² - (\frac{a}{2})²$};

\onslide<1->
  % Чворови за променљиве
  \node[draw, circle] (VO) at (3,4) {$O$};
  \node[draw, circle] (VP) at (3,3) {$P$};
  \node[draw, circle] (Va) at (3,2) {$a$};
  \node[draw, circle] (Vb) at (3,1) {$b$};
  \node[draw, circle] (Vh) at (3,0) {$h$};

  % Гране између чворова
  \draw (FO) -- (VO);
  \draw (FO) -- (Va);
  \draw (FO) -- (Vb);
  \draw (FP) -- (VP);
  \draw (FP) -- (Va);
  \draw (FP) -- (Vh);
  \draw (Fh) -- (Vh);
  \draw (Fh) -- (Va);
  \draw (Fh) -- (Vb);
\end{tikzpicture}
\hfill
\begin{tikzpicture}
\onslide<1->
  % Чворови за формуле
  \node[draw, rectangle] (FO) at (0,5) {$O = 3a$};
  \node[draw, rectangle] (FPa) at (0,4) {$P = \frac{a² \sqrt{3}}{4}$};
  \node[draw, rectangle] (Fh) at (0,3) {$h = \frac{a \sqrt{3}}{2}$};
  \node[draw, rectangle] (FP) at (0,2) {$P = \frac{ah}{2}$};
  \node[draw, rectangle] (Fro) at (0,1) {$r_o = \frac 23 h$};
  \node[draw, rectangle] (Fru) at (0,0) {$r_u = \frac 13 h$};

\onslide<1->
  % Чворови за променљиве
  \node[draw, circle] (VO) at (3,5) {$O$};
  \node[draw, circle] (Va) at (3,4) {$a$};
  \node[draw, circle] (VP) at (3,3) {$P$};
  \node[draw, circle] (Vh) at (3,2) {$h$};
  \node[draw, circle] (Vro) at (3,1) {$r_o$};
  \node[draw, circle] (Vru) at (3,0) {$r_u$};

  % Гране између чворова
  \draw (FO) -- (VO);
  \draw (FO) -- (Va);
  \draw (FPa) -- (VP);
  \draw (FPa) -- (Va);
  \draw (Fh) -- (Vh);
  \draw (Fh) -- (Va);
  \draw (FP) -- (VP);
  \draw (FP) -- (Va);
  \draw (FP) -- (Vh);
  \draw (Fro) -- (Vro);
  \draw (Fro) -- (Vh);
  \draw (Fru) -- (Vru);
  \draw (Fru) -- (Vh);
\end{tikzpicture}
\end{frame}

\begin{frame}{Ромб}
\begin{tikzpicture}
\onslide<1->
  % Чворови за формуле
  \node[draw, rectangle] (FO) at (0,6) {$O = 4a$};
  \node[draw, rectangle] (FPa) at (0,5) {$P = ah$};
  \node[draw, rectangle] (FP) at (0,4) {$P = \frac{d_1 \cdot d_2}{2}$};
  \node[draw, rectangle] (Fa) at (0,2) {$a² = (\frac{d_1}{2})² + (\frac{d_2}{2})²$};
  \node[draw, rectangle] (Fru) at (0,0) {$r_u = \frac h2$};

\onslide<1->
  % Чворови за променљиве
  \node[draw, circle] (VO) at (4,6) {$O$};
  \node[draw, circle] (VP) at (4,5) {$P$};
  \node[draw, circle] (Va) at (4,4) {$a$};
  \node[draw, circle] (Vh) at (4,3) {$h$};
  \node[draw, circle] (Vd1) at (4,2) {$d_1$};
  \node[draw, circle] (Vd2) at (4,1) {$d_2$};
  \node[draw, circle] (Vru) at (4,0) {$r_u$};

  % Гране између чворова
  \draw (FO) -- (VO);
  \draw (FO) -- (Va);
  \draw (FPa) -- (VP);
  \draw (FPa) -- (Va);
  \draw (FPa) -- (Vh);
  \draw (FP) -- (VP);
  \draw (FP) -- (Vd1);
  \draw (FP) -- (Vd2);
  \draw (Fa) -- (Va);
  \draw (Fa) -- (Vd1);
  \draw (Fa) -- (Vd2);
  \draw (Fru) -- (Vru);
  \draw (Fru) -- (Vh);
\end{tikzpicture}
\end{frame}

\begin{frame}{Правоугли и једнакокраки трапез}
\begin{tikzpicture}
\onslide<1->
  % Чворови за формуле
  \node[draw, rectangle] (FO) at (0,6) {$O = a+b+c+h$};
  \node[draw, rectangle] (FP) at (0,4) {$P = mh$};
  \node[draw, rectangle] (Fm) at (0,2) {$m = \frac{a+b}{2}$};
  \node[draw, rectangle] (Fh) at (0,0) {$h² = c² - (a-b)²$};

\onslide<1->
  % Чворови за променљиве
  \node[draw, circle] (VO) at (3,6) {$O$};
  \node[draw, circle] (VP) at (3,5) {$P$};
  \node[draw, circle] (Va) at (3,4) {$a$};
  \node[draw, circle] (Vb) at (3,3) {$b$};
  \node[draw, circle] (Vc) at (3,2) {$c$};
  \node[draw, circle] (Vm) at (3,1) {$m$};
  \node[draw, circle] (Vh) at (3,0) {$h$};

  % Гране између чворова
  \draw (FO) -- (VO);
  \draw (FO) -- (Va);
  \draw (FO) -- (Vb);
  \draw (FO) -- (Vc);
  \draw (FO) -- (Vh);
  \draw (FP) -- (VP);
  \draw (FP) -- (Vm);
  \draw (FP) -- (Vh);
  \draw (Fm) -- (Vm);
  \draw (Fm) -- (Va);
  \draw (Fm) -- (Vb);
  \draw (Fh) -- (Vh);
  \draw (Fh) -- (Vc);
  \draw (Fh) -- (Va);
  \draw (Fh) -- (Vb);
\end{tikzpicture}
\hfill
\begin{tikzpicture}
\onslide<1->
  % Чворови за формуле
  \node[draw, rectangle] (FO) at (0,6) {$O = a+b+2c$};
  \node[draw, rectangle] (FP) at (0,4) {$P = mh$};
  \node[draw, rectangle] (Fm) at (0,2) {$m = \frac{a+b}{2}$};
  \node[draw, rectangle] (Fh) at (0,0) {$h² = c² - (\frac{a-b}{2})²$};

\onslide<1->
  % Чворови за променљиве
  \node[draw, circle] (VO) at (3,6) {$O$};
  \node[draw, circle] (VP) at (3,5) {$P$};
  \node[draw, circle] (Va) at (3,4) {$a$};
  \node[draw, circle] (Vb) at (3,3) {$b$};
  \node[draw, circle] (Vc) at (3,2) {$c$};
  \node[draw, circle] (Vm) at (3,1) {$m$};
  \node[draw, circle] (Vh) at (3,0) {$h$};

  % Гране између чворова
  \draw (FO) -- (VO);
  \draw (FO) -- (Va);
  \draw (FO) -- (Vb);
  \draw (FO) -- (Vc);
  \draw (FP) -- (VP);
  \draw (FP) -- (Vm);
  \draw (FP) -- (Vh);
  \draw (Fm) -- (Vm);
  \draw (Fm) -- (Va);
  \draw (Fm) -- (Vb);
  \draw (Fh) -- (Vh);
  \draw (Fh) -- (Vc);
  \draw (Fh) -- (Va);
  \draw (Fh) -- (Vb);
\end{tikzpicture}
\end{frame}

\end{document}

\begin{frame}{Правоугаоник}
\begin{forest}
  for tree={
    draw,
    rectangle,
    minimum size=1em,
    inner sep=2pt,
    parent anchor=south,
    child anchor=north,
    l sep=5mm,
    s sep=5mm,
  }
  [{Правоугаоник}
  [{$O = 2a + 2b$}
    [{$O$}]
    [{$a$}]
    [{$b$}]
  ]
  [{$P = ab$}
    [{$P$}]
    [{$a$}]
    [{$b$}]
  ]
  [{$d² = a² + b²$}
    [{$d$}]
    [{$a$}]
    [{$b$}]
  ]
  [{$r_o = d/2$}
    [{$r_o$}]
    [{$d$}]
  ]]
\end{forest}
\end{frame}

\begin{frame}{Правоугаоник}
\begin{forest}
  for tree={
    draw,
    rectangle,
    parent anchor=south,
    child anchor=north,
    l sep=1cm,
  }
  [
    [{$O = 2a + 2b$}, tikz={\node[rectangle] at (.5,.5) {$O$};} ]
    [{$P = ab$}, tikz={\node[rectangle] at (.5,.5) {$P$};} ]
    [{$d^2 = a^2 + b^2$}, tikz={\node[rectangle] at (.5,.5) {$d^2$};} ]
    [{$r_o = \frac{d}{2}$}, tikz={\node[rectangle] at (.5,.5) {$r_o$};} ]
  ]
  [
    [{$O$}, tikz={\node[circle] at (.5,.5) {$O$};} ]
    [{$P$}, tikz={\node[circle] at (.5,.5) {$P$};} ]
    [{$a$}, tikz={\node[circle] at (.5,.5) {$a$};} ]
    [{$b$}, tikz={\node[circle] at (.5,.5) {$b$};} ]
    [{$d$}, tikz={\node[circle] at (.5,.5) {$d$};} ]
    [{$r_o$}, tikz={\node[circle] at (.5,.5) {$r_o$};} ]
  ]
\end{forest}
\end{frame}

\begin{frame}{Бипартитни граф}
    \begin{tikzpicture}[grow=right,sibling distance=6em,level distance=10em]
%        \graph [layered layout] {
        \graph {
            subgraph cluster {
                node [rectangle,draw] {"Питагорина теорема"};
                node [rectangle,draw] {"Обим правоугаоника"};
                node [rectangle,draw] {"Површина правоугаоника"};
                node [rectangle,draw] {"Полупречник описаног круга"};
            };
            subgraph cluster {
                node [circle,draw] {$a$};
                node [circle,draw] {$b$};
                node [circle,draw] {$c$};
            };
            {"Питагорина теорема"} -- {$a$};
            {"Питагорина теорема"} -- {$b$};
            {"Питагорина теорема"} -- {$c$};
            % ... остале везе
        };
    \end{tikzpicture}
\end{frame}

\begin{frame}{Povezanost formula i promenljivih}
    \begin{tikzpicture}[>=latex]
        \graph[layered layout, sibling distance=2cm, level distance=3cm] {
            "Pitagorina teorema" -> {"a^2 + b^2 = c^2"};
            "Obim pravouglog trougla" -> {"a + b + c"};
            % Dodajte ostale formule i promenljive
        };
    \end{tikzpicture}
\end{frame}

\begin{frame}%{Правоугаоник}
\usetikzlibrary {graphs}
\tikz [>={To[sep]}, rotate=90, xscale=-1,
       mark/.style={fill=black!50}, mark/.default=]
  \graph [trie, simple,
          nodes={circle,draw},
          edges={nodes={
              inner sep=1pt, anchor=mid,
              fill=graphicbackground}}, % yellowish background
          put node text on incoming edges]
    {
      root[mark] -> {
        a -> n -> {
          g [mark],
          f -> a -> n -> g [mark]
        },
        f -> a -> n -> g [mark],
        g[mark],
        n -> {
          g[mark],
          f -> a -> n -> g[mark]
        }
      },
      { [edges=red] % highlight one path
        root -> f -> a -> n
      }
    };
\end{frame}

\begin{frame}%{Правоугаоник}
    \begin{tikzpicture}[scale=1.5]
        \draw (0,0) rectangle (4,2);
        \draw (0,0) -- (4,2);
        \node at (2,1) {$a$};
        \node at (1,0) {$b$};
        \node at (2.5,2.2) {$c$};
        \node at (3,1) {$d$};
        \node at (2,-0.5) {O};
        \draw (2,0) circle (1cm);
        
%        \node at (6,1) {Формуле:};
        \node at (6,0.5) {$c^2 = a^2 + b^2$};
        \node at (6,0) {$O = 2a + 2b$};
        \node at (6,-0.5) {$P = ab$};
        \node at (6,-1) {$r = \frac{c}{2}$};
    \end{tikzpicture}
\end{frame}
