\documentclass[12pt]{beamer}
\usetheme{Copenhagen}
\usecolortheme{beaver}
\usefonttheme[onlymath]{serif}

\usepackage[OT2]{fontenc}
\usepackage[utf8x]{inputenc}
\usepackage[serbian]{babel}
\usepackage{amsmath,amsfonts,amssymb}
\usepackage{graphicx,tikz}

\author{Аутор}
\title{Геометријске илустрације}
%\setbeamercovered{transparent} 
%\setbeamertemplate{navigation symbols}{} 
%\logo{} 
\institute{ОШ „Иван Горан Ковачић“, Станишић}
\date{12. децембар 2024.}
\subject{Математика 6}

\begin{document}

\begin{frame}
\titlepage
\end{frame}

%\begin{frame}
%\tableofcontents
%\end{frame}

\begin{frame}{Ромб}
Примена Питагорине теореме на ромб:
$$
	a^2 =
	\left( \frac{d_1}{2} \right)^2 +
	\left( \frac{d_2}{2} \right)^2
$$
\end{frame}

\begin{frame}{Правоугли троугао}
\begin{tikzpicture}
  \coordinate (C) at (0,0);
  \coordinate (A) at (4,0);
  \coordinate (B) at (0,3);

  \draw (A)
  	-- (B) node[midway, above right] {$c = 5 \,\mathrm{cm}$}
  	-- (C) node[midway, left] {$a = 3 \,\mathrm{cm}$}
  	-- cycle node[midway, below] {$b = 4 \,\mathrm{cm}$};

  \draw (A) circle (1pt) node[below right] {$A$};
  \draw (B) circle (1pt) node[above left] {$B$};
  \draw (C) circle (1pt) node[below left] {$C$};
\end{tikzpicture}
\end{frame}

\begin{frame}{Правоугаоник и кружница}
\begin{tikzpicture}
  \draw (0,0) circle (1cm);
  \draw (3,0) rectangle (5,2);
\end{tikzpicture}
\end{frame}

\end{document}
